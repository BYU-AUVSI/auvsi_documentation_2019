% !TEX root=main.tex

We developed a system requirements matrix in conjunction with the AUVSI competition rules (see artifact RM-001). All system-wide performance measures were considered, and five measures listed in Table~\ref{tab:key_measures} were selected as key success measures. Over the course of the next two semesters, we will gauge the desirability of our product based on how well the product completes each of these performance measures. Each performance measure will be evaluated in an environment designed to mimic the competition.

\begin{table}[H]
	\centering
	\caption{Key success measures for the UAS}\label{tab:key_measures}
\begin{tabular}{|P{3cm}|P{1.8cm}|P{1.7cm}|P{1.2cm}|P{1cm}|P{1.7cm}|P{1.2cm}|P{1.7cm}|}
	\hline
\rowcolor[HTML]{C0C0C0}	
	{\color[HTML]{000000} \textbf{Measures (units)}} & {\color[HTML]{000000}\textbf{Stretch Goal}} & {\color[HTML]{000000}\textbf{Excellent (A)}} & {\color[HTML]{000000}\textbf{Good (B)}} & {\color[HTML]{000000}\textbf{Fair (C)}} & {\color[HTML]{000000}\textbf{Lower Acceptable}} & {\color[HTML]{000000}\textbf{Ideal}} & {\color[HTML]{000000}\textbf{Upper Acceptable}} \\
	\hline
	\textbf{Obstacles Hit (\#)} & 0 & 1 & 3 & 5 & 0 & 0 & 5 \\
	\hline
	\textbf{Average Waypoint Proximity (ft)*} & 5 & 20 & 25 & 30 & 0 & 0 & 100 \\
	\hline
	\textbf{Characteristics Identified (\%)**} & 80 & 40 & 30 & 20 & 20 & 100 & 100 \\
	\hline
	\textbf{Airdrop Accuracy (ft)} & 5 & 25 & 50 & 75 & 0 & 0 & 75 \\
	\hline
	\textbf{Number of Manual Takeovers} & 0 & 1 & 2 & 3 & 0 & 0 & 3 \\
	\hline
\end{tabular}
\end{table}

* \textit{Average Waypoint Proximity} refers to the norm of the distance between the UAS and the waypoint location at the point when the autopilot considers the waypoint to be captured.

** \textit{Characteristics Identified} refers to the ability to classify the color, shape, and textual content of visual targets scattered on the ground using camera measurements.

% section description found in page 145 of the project book, with an example on page 146
