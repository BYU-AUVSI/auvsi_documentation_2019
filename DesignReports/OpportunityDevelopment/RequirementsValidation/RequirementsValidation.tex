\documentclass[]{auvsi_doc}
\setkeys{auvsi_doc.cls}{
	AUVSITitle={Requirements Validation},
	AUVSIRevision=0.0,
	AUVSIDescription={Initial Draft},
	AUVSIAuthor={Kameron Eves},
	AUVSIChecker={[DOCUMENT CHECKER]},
	AUVSILogoPath={./figs/logo.pdf},
	AUVSIDocID={DJ-005}
}

% include extra packages, if needed

% Remove Heading Numbers
\setcounter{secnumdepth}{0}

\begin{document}
\CapstoneTitlePage
% document contents

\section{Purpose}
\section{Market Representatives}
The market for this product consists of the AUVSI SUAS Competition judges.
They will determine the ultimate success of the product through its performance in the AUVSI SUAS Competition.
These judges provide a set of competition requirements that are our primary source for market requirements.
On September 14th, 2018 the competition requirements for the 2019 competition year were released.
From September 14th, 2018 to October 4th, 2018 the competition judges provide a comment period where team members request clarification on the requirements.
The current competition requirements and the comment period are our primary means of determining the market requirements.
In addition, our coach, Dr. Ning, and our sponsor, Dr. McLain, provide on-campus market representation, and feedback on our capstone documents. 

\section{Development of the Opportunity}

To determine the market requirements we started with a detailed study of the AUVSI SUAS competition requirements.
We determined the most crucial portions of the competition requirements by benchmarking individual team's performance in previous competitions (recorded in DJ-003), and by breaking down the relative scoring weights of the competition requirements (recorded in DJ-004). 
Using these studies, we prepared a set of requirements ready for feedback from our on campus market representatives.
On October 1st, 2018, we presented the requirements to Dr. Ning and Dr. McLain.
The feedback we received was positive, and encouraged us to refine the requirements through developing subsystem requirements in the future.

\section{Justification and Validation of Key Success Measures}
Using the information obtained from the market representatives, we created a list of key success measures which we will use to as a measuring stick to validate our product. We feel strongly that if we do not achieve at least fair performance in all of these key success measures, then our product will be unsatisfactory and we will consider the project a failure. This is primarily because the UAS would not meet the market's needs. On the other hand, we feel equally as strongly that if we achieve excellent performance in most, or preferably all, of these key success measures, then our product will represent outstanding work that precisely fulfills the needs of the market. What follows is an enumeration and justification of our the key success measures. 

Because our product is inherently for a competition, most of these key success measures use the distribution of points we can obtain as a frame work. A break down of the point distribution can be found in DJ-004. Another important source for these decisions was the performance of last years BYU team. This helped us determine what was possible and what isn't possible considering the resources available and th

\begin{itemize}
\item \textbf{Obstacles Hit} This constitutes 20\% of the points our UAS can obtain. This year, the judges have increased the number of obstacles that need to be avoided from 20 to 30. As such, we felt this an important focus of our efforts. Last years team hit 3 obstacles and only 48\% of all teams in the competition were able to avoid any obstacles. 
	\begin{itemize}
	\item \underline{Stretch: 0 Obstacles} - The ideal is to avoid all obstacles. While this is difficult, we do feel it is possible and so have set this as our stretch goal.
	\item \underline{Excellent: 1 Obstacle} -  Accounting for the increased number of obstacles, decreasing the number of obstacles hit from 3 to 1 represents a marked improvement from last year. As such we have set this as excellent performance.
	\item \underline{Good: 3 Obstacles} -  As in all aspects of the competition, we hope to improve upon last years performance. Therefore, we feel that avoiding 3 obstacles (the same number as last year) constitutes merely a good performance. Again accounting for the increased number of obstacles, this would be an improvement from last year.
	\item  \underline{Fair: 5 Obstacles} -  If last years performance is scaled for the increased number of obstacles (a 50\% increase) we would need to avoid 5 obstacles to match last years performance. It is likely that the number of obstacles hit increases exponentially with the number of obstacles in the competition. This is because the likelihood of avoiding obstacles by luck decreases exponentially. Therefore, avoiding 5 obstacles might be numerically equivalent to last years performance, but could still indicate some small improvement. Thus we choose 5 obstacles as a fair performance. Anything less then 5 would indicate a failure to improve last years system.
	\end{itemize}
\item \textbf{Waypoint Proximity} Autonomous flight of a waypoint path constitutes 20\% of the points our UAS can obtain. So this is a significant part of the competition. Among other things, points are awarded for how close the UAS comes to each waypoint. Last years team averaged 16 feet from the waypoints. Only 56\% of all teams in the competition were even able to get points for flying a waypoint path.
	\begin{itemize}
	\item \underline{Stretch: 10 feet} - 
	\item \underline{Excellent: 15 feet} -  
	\item \underline{Good: 20 feet} -  
	\item  \underline{Fair: 25 feet} -  
	\end{itemize}
\item \textbf{Objects Identified} Autonomous flight of a waypoint path constitutes 20\% of the points our UAS can obtain. So this is a significant part of the competition. Among other things, points are awarded for how close the UAS comes to each waypoint. Last years team averaged 16 feet from the waypoints. Only 56\% of all teams in the competition were even able to get points for flying a waypoint path.
	\begin{itemize}
	\item \underline{Stretch: Autonomous Detection} - 
	\item \underline{Excellent: 40\%} -  
	\item \underline{Good: 30\%} -  
	\item  \underline{Fair: 20\%} -  
	\end{itemize}
\item \textbf{Airdrop Accuracy} Autonomous flight of a waypoint path constitutes 20\% of the points our UAS can obtain. So this is a significant part of the competition. Among other things, points are awarded for how close the UAS comes to each waypoint. Last years team averaged 16 feet from the waypoints. Only 56\% of all teams in the competition were even able to get points for flying a waypoint path.
	\begin{itemize}
	\item \underline{Stretch: 5 feet} - 
	\item \underline{Excellent: 25 feet} -  
	\item \underline{Good: 50 feet} -  
	\item  \underline{Fair: 75 feet} -  
	\end{itemize}
\item \textbf{Number of Manual Takovers} Autonomous flight of a waypoint path constitutes 20\% of the points our UAS can obtain. So this is a significant part of the competition. Among other things, points are awarded for how close the UAS comes to each waypoint. Last years team averaged 16 feet from the waypoints. Only 56\% of all teams in the competition were even able to get points for flying a waypoint path.
	\begin{itemize}
	\item \underline{Stretch: 0 Takeover} - 
	\item \underline{Excellent: 1 Takeovers} -  
	\item \underline{Good: 2 Takeovers} -  
	\item  \underline{Fair: 3 Takeover} -  
	\end{itemize}

\end{itemize}


\section{Validation of the Completed Sections of the Requirements Matrix}


\end{document}
