% Header
\documentclass[]{auvsi_doc}
\setkeys{auvsi_doc.cls}{
	AUVSITitle={Benchmarking Artifact},
	AUVSIRevision=0.1,
	AUVSIDescription={Initial Draft},
	AUVSIAuthor={Kameron Eves},
	AUVSIChecker={[DOCUMENT CHECKER]},
	AUVSILogoPath={./figs/logo.pdf},
	AUVSIDocID={DJ-003}
}

\usepackage{multirow}
\usepackage{adjustbox}
\usepackage{color}

\definecolor{dg}{RGB}{0,100,0}

% Remove Heading Numbers
\setcounter{secnumdepth}{0}

\begin{document}

\CapstoneTitlePage

\section{Introduction}

This project is inherently a competition. Therefore, how the product compares against other teams is an important metric.  A quick search of previous years shows that the same several teams tend to finish at the top, only swapping places among themselves. For this reason, it is valuable to observe these top 5 teams' performances. At this stage of the product development process, the main focus of this external examination is to see where we should focus our time. If we can find where last year's BYU's team lacked but the best teams performed well then we can have a suggestion as to where to focus our efforts.

\section{Method}
The AUVSI-SUAS administration publishes copies of each team's final report. From these reports we were able to parse what the teams attempted. The competition organizers also distributed a summary of the points for each team. From this we found what each team successfully achieved. This information is tabulated in Table~\ref{table:benchmark}

\begin{table}[h!]
	
	\caption{Last years performance of the top 5 teams. Note that if the table indicates that a team "Achieved" something, it only indicates that they got some points for that task - not that they were 100\% successful.}
	\centering
	\begin{tabular}{ cllccccccc }
		\multicolumn{9}{c}{\textbf{\underline{Key}}} \\
		\multicolumn{4}{l}{\textbf{AF =} Autonomous Flight} & \multicolumn{5}{l}{\textbf{OA =} Obstacle Avoidance} \\
		\multicolumn{4}{l}{\textbf{AOD =} Autonomous Object Detection} & \multicolumn{5}{l}{\textbf{OC =} Object Classification} \\
		\multicolumn{4}{l}{\textbf{OL =} Object Localization} & \multicolumn{5}{l}{\textbf{AF =} Payload Delivery} \\
		\hline
		\textbf{Rank} 	& \textbf{Team} 	&  	& \textbf{AF}  	& \textbf{OA} 	& \textbf{AOD}  	& \textbf{OC} 	& \textbf{OL} 	& \textbf{PD} \\
		\hline
		\multirow{2}{*}{1} 		& \multirow{2}{*}{UdeS} 				& Tried 		& \color{dg}\textbf{Y } 	& \color{dg}\textbf{Y}  	& \color{dg}\textbf{Y} 	& \color{dg}\textbf{Y} 	& \color{dg}\textbf{Y}  	& \color{dg}\textbf{Y}  \\
	 	  					&  								& Achieved	& \color{dg}\textbf{Y} 	& \color{dg}\textbf{Y}  	& \color{dg}\textbf{Y} 	& \color{dg}\textbf{Y} 	& \color{dg}\textbf{Y}  	& \color{dg}\textbf{Y}  \\
		\hline
		\multirow{2}{*}{2} 		& \multirow{2}{*}{Flint Hill School} 		& Tried		& \color{dg}\textbf{Y} 	& \color{dg}\textbf{Y}  	& \color{dg}\textbf{Y} 	& \color{dg}\textbf{Y} 	& \color{dg}\textbf{Y}  	& \color{dg}\textbf{Y}  \\
	 	  					&  								& Achieved	& \color{dg}\textbf{Y} 	& \color{dg}\textbf{Y}  	& \color{dg}\textbf{Y} 	& \color{dg}\textbf{Y} 	& \color{dg}\textbf{Y}  	& \color{dg}\textbf{Y}  \\		
		\hline
		\multirow{2}{*}{3} 		& \multirow{2}{*}{VT \& VSU} 			& Tried		& \color{dg}\textbf{Y} 	& \color{dg}\textbf{Y}  	& \color{red}\textbf{N} 	& \color{dg}\textbf{Y} 	& \color{dg}\textbf{Y}  	& \color{dg}\textbf{Y}  \\	 	  							&  								& Achieved	& \color{dg}\textbf{Y} 	& \color{dg}\textbf{Y}  	& \color{red}\textbf{N} 	& \color{red}\textbf{N} 	& \color{red}\textbf{N}  	& \color{dg}\textbf{Y}  \\		
		\hline
		\multirow{2}{*}{4} 		& \multirow{2}{*}{Cornell} 				& Tried		& \color{dg}\textbf{Y} 	& \color{dg}\textbf{Y}  	& \color{dg}\textbf{Y} 	& \color{dg}\textbf{Y} 	& \color{dg}\textbf{Y}  	& \color{dg}\textbf{Y}  \\	 	  	
							&  								& Achieved	& \color{dg}\textbf{Y} 	& \color{dg}\textbf{Y}  	& \color{dg}\textbf{Y} 	& \color{dg}\textbf{Y} 	& \color{dg}\textbf{Y}  	& \color{dg}\textbf{Y}  \\		
		\hline
		\multirow{2}{*}{5} 		& \multirow{2}{*}{MPSTME \& NMIMS} 	& Tried		& \color{dg}\textbf{Y} 	& \color{dg}\textbf{Y}  	& \color{red}\textbf{N} 	& \color{red}\textbf{N} 	& \color{red}\textbf{N}  	& \color{dg}\textbf{Y}  \\
							&  							   	& Achieved	& \color{dg}\textbf{Y} 	& \color{dg}\textbf{Y}  	& \color{red}\textbf{N} 	& \color{red}\textbf{N} 	& \color{red}\textbf{N}  	& \color{dg}\textbf{Y}  \\		
		\hline
		\multirow{2}{*}{9} 		& \multirow{2}{*}{BYU} 				& Tried		& \color{dg}\textbf{Y} 	& \color{dg}\textbf{Y}  	& \color{red}\textbf{N} 	& \color{dg}\textbf{Y} 	& \color{dg}\textbf{Y}  	& \color{dg}\textbf{Y}  \\
							&  								& Achieved	& \color{dg}\textbf{Y} 	& \color{dg}\textbf{Y}  	& \color{red}\textbf{N} 	& \color{dg}\textbf{Y} 	& \color{red}\textbf{N}  	& \color{red}\textbf{N}  \\
		\hline
	\end{tabular}
	\label{table:benchmark}
\end{table}

\section{Results}

As can be seen in Table~\ref{table:benchmark}, the teams that won the competition attempted and succeeded at all of the tasks. However, the table indicates that when a top team does not succeed at something, it is in the object detection section of the competition. One team, MPSTME \& NMIMS, did not even attempt this portion of the competition. VT \& VSU did not successfully achieve object detection despite the fact that they attempted to do so manually (the easier method) rather then autonomously. However, three of the five best best teams did successfully achieve autonomous object detection. For this reason, we feel that autonomous object detection is the right thing to pursue. Autonomous object detection is a difficult task. Thus, for reasons of reliability, we will also manually detect the objects. The competition allows for both methods to be employed. Points are only awarded for the method that results in the highest final score.

Also of note in Table~\ref{table:benchmark} is that every team in the top five attempted and achieved payload drop. Last years BYU team did attempt payload drop, but did not achieve it due to unrelated technical issues. This will be the place were the highest improvement to cost ratio can be obtained. This is especially true because the percentage of competition points awarded for the payload drop has increased. Therefore, we feel that focusing on the payload drop will be a good use of time.

\section{Discussion}

The current system upon which we are iterating was successful at autonomous flight and obstacle avoidance. These system components are designed such that they will continue to work well for this years competition. However, as shown above, effort must be put into the payload delivery and autonomous obstacle detection. Because there is not currently an autonomous object detection system and because the payload delivery is significantly more complicated then previous years, these two tasks will consume most of our time. We have divided our teams into two sub teams, one for the payload delivery and one for. By focusing on these two tasks we feel that our performance in the competition will rise to and even exceed the performance of the top teams.



\end{document}
