\documentclass[]{auvsi_doc}
\setkeys{auvsi_doc.cls}{
	AUVSITitle={Advice from 2019 AUVSI Team},
	AUVSIRevision=0.0,
	AUVSIDescription={Created},
	AUVSIAuthor={Derek Knowles},
	AUVSIChecker={[Checker]},
	AUVSILogoPath={./figs/logo.pdf},
	AUVSIDocID={AF-012}
}

% include extra packages, if needed
\usepackage{makecell}
\usepackage{longtable}
\usepackage{hyperref}
\usepackage{booktabs}

% Remove Heading Numbers
\setcounter{secnumdepth}{0}
\providecommand{\tightlist}{%
  \setlength{\itemsep}{0pt}\setlength{\parskip}{0pt}}

\begin{document}
\begin{AUVSITitlePage}
\begin{artifacttable}
	\entry{PS-002, 1.0, 7-29-2018, Converted from Google Doc, Jacob Willis, Jake Johnson}
\end{artifacttable}
\end{AUVSITitlePage}
% document contents
\section{Introduction}
This document has been made to help the AUVSI 2019-20 team have a strong start and provide them with advice and knowledge about their potential resources. 
It is a loosely organized list of advice, as well as contact information for everyone who was previously on the team.

Best of luck!
\textit{AUVSI 2018-2019 Team}



\section{Design and Development}
{``This is a very complex systems engineering project--more so (in my
opinion) than almost every other Capstone project. Don't rely on the
pacing that the Capstone class demands; demand more of yourselves and
don't make excuses for not getting the plane up in the air. Look into
different project management techniques that emphasize iterative testing
and agile development. It all seems easier at first than it ends up
being, so don't rest on your laurels, even if you already have
experience with ROS, ROSplane, ROSflight, aircraft design, etc.'' -
Andrew Torgesen}

{}

{``As soon as the draft rules come out for the competition
(August-September) I would check the point percentages and concentrate
your time on the critical aspects of the competition that you need in
order to compete. Only after you have those critical pieces working well
I'd add additional bonus functionality.'' - Derek Knowles}

{}

{``Holding mock competitions early will be invaluable in your
preparation'' - Ryan Anderson}

{}

{``Make your flight test time count. This is done primarily by testing
your subsystem as best you can beforehand in the lab so you're really
ready to test in the air. We spent an inordinate amount of time setting
up networks and trying to get software running correctly, while most of
these wrinkles could have been ironed out in the lawn in front of the
EB, and some of them even in the lab. We found that a ``mock flight
test'' where we set up everything outside (where we had gps signal) but
didn't actually fly was extremely helpful and required negligible travel
time.'' -Ryan Anderson}

{}

{``Keep and follow a checklist. There are way too many details for one
person to keep straight 100\% of the time. If something unexpected
happens, add it to the checklist. It may seem unnecessary and
monotonous, but it saved us from many crashes.'' -Ryan Anderson}

{}

{``When you run into an issue during flight testing, write down what the
cause was, as well as the solution in a place where all team members can
view it. Chances are that the issue will resurface down the line, and
everyone should have access to information for fixing the problem,
should the need arise.'' - Andrew Torgesen}

{}

{``Do unit testing when you add new features; the imaging team was a
great example of how to do that right this year, and their software was
more mature and reliable in the end.'' - Andrew Torgesen}

{}

{``It is 100\% worth your time to learn good software practices, and
then use them. The vast majority of your work will be software
development. I'm totally willing to come during capstone time every
other Friday and talk about git, unit testing, continuous integration,
etc as needed.'' -Tyler Miller}

{}

{``I think what would help with handing off from year to year is getting
help from underclassmen. Many people are willing and want to help. They
would learn a lot and could be on the future team. Then there wouldn't
be a blind hand off every year. If there is good leadership and
delegation, having more help will not hold you back. They can work on
the non crucial tasks, so you aren't depending on their success. But
they would learn a lot in the process and be in the loop.'' -Brady Moon}

{}

{``Consider dependencies between subsystems when planning schedule - for
example the UGV drop relies on the full autopilot/planning stack before
it can be tested.'' - Jacob Willis}

{}

{``Find ways to maximize the amount of time spent flying autonomously on
flight tests. It takes a lot of time to travel to/from good testing
areas, so make sure that the time there is very effective.'' - Jacob
Willis}

{}

{``Find a good, repository friendly way to share code between the flight
computer and personal computers without an internet connection. I think
it should be possible to set the BRIX as an alternate remote for the
odroid - that way git commits can be communicated without an internet
connection.'' - Jacob Willis}

{}

{``Debug the simulator! Get everyone set up to run the full system on
their laptops - this will help when gaps need to be filled in.'' - Jacob
Willis}

{}

{``We had problems with the Ubiqiti until we turned off the 5GHz wifi on
the wifi router. There are some guis you can access at the IP address of
the lightbeam and bullet. These are useful for debugging network
issues.'' - Jacob Willis}

{}

{``When doing mock competitions (or testing individual subsystems), make
sure to keep track of your score. Imagining what you would have scored
is not nearly as useful to you as figuring out exactly what you would
have scored and recording it somewhere. This will give you a clear
measuring stick to base your progress on.'' - Tyler Critchfield}

{}

{``I would make sure that by the end of the school year, the team has
done multiple mock competitions (with everyone participating in their
anticipated roles), while timing everything just like the real thing.
After finals end, people will leave for internships or graduate school
and you'll likely have less time and manpower than you anticipate.'' -
Tyler Critchfield}

{}

{``In the early months of both semesters (Sep-Oct, Jan-Feb) you will
likely have more time to devote to flight tests because your other
classes will not be as busy yet. Take advantage of that time if you can,
especially in the fall when the weather is nice.'' - Tyler Critchfield}


{``Flight test locations pros/cons:}

\begin{itemize}
\tightlist
\item
  {Rock Canyon Park: }
\end{itemize}

\begin{itemize}
\tightlist
\item
  {close and quick to reach}
\item
  {has power and a covered pavilion}
\item
  {does not have much area to fly in and is near houses}
\item
  {There seems to be a downdraft coming out of the canyon (southeast
  corner of the park) that caused one catastrophic crash and almost
  another}
\item
  {Good for quick RC tests, not for autopilot - if possible avoid flying
  here due to safety concerns}
\end{itemize}

\begin{itemize}
\tightlist
\item
  {Springville (off the freeway)}
\end{itemize}

\begin{itemize}
\tightlist
\item
  {Lots of space (just stay away from the buildings and freeway behind
  you)}
\item
  {Close-by}
\item
  {We had lots of connection issues that led to crashes}
\item
  {Technically within 5-miles of the Spanish Fork airport}
\item
  {We recommend no flying here just from our experience and proximity to
  the airport.}
\end{itemize}

\begin{itemize}
\tightlist
\item
  {Lehi Airspace
  (}{\href{https://www.google.com/url?q=https://www.google.com/maps/place/Utah\%2BCounty\%2BRadio\%2BControlled\%2BAirplane\%2BPark/@40.3637057,-111.900469,701m/data\%3D!3m1!1e3!4m8!1m2!2m1!1slehi\%2Brc\%2Bairspace!3m4!1s0x874d7f21d54370cd:0xf56ece4cdeef437d!8m2!3d40.3634498!4d-111.9019458\&sa=D\&ust=1564438983443000}{https://www.google.com/maps/place/Utah+County+Radio+Controlled+Airplane+Park/@40.3637057,-111.900469,701m/data=!3m1!1e3!4m8!1m2!2m1!1slehi+rc+airspace!3m4!1s0x874d7f21d54370cd:0xf56ece4cdeef437d!8m2!3d40.3634498!4d-111.9019458}}{)}
\end{itemize}

\begin{itemize}
\tightlist
\item
  {No power}
\item
  {No car access unless someone else is already there}
\end{itemize}

\begin{itemize}
\tightlist
\item
  {If the gate is locked you can walk everything in if you have a
  generator. We once did it with power from a car, but the extension
  cord is only so long and it was a weird set-up all together}
\end{itemize}

\begin{itemize}
\tightlist
\item
  {More space than Rock Canyon, but still not a lot}
\item
  {Be careful of a river which would be bad to crash in (also lots of
  reeds, planes get lost in those frequently)}
\item
  {Nice area to take-off and land, with grass and asphalt (more similar
  to competition environment)}
\item
  {Often is frequented by other RC enthusiasts. Two older gentlemen
  seemed to be there pretty much every morning. Pretty soon they didn't
  really like us being there taking up their time and space, and they
  let us know that there were rules we were breaking. One is that the
  pilot needs to have some sort of insurance. If you fly there, I think
  it's around \$20 or so, and could easily be covered by the budget.
  They also probably thought we were to blame for one of their crashes
  (even though at the time our system wasn't even on). }
\item
  {We decided it wasn't worth the hassle to continue going there, but
  could be an option if Goshen doesn't work for some reason. }
\end{itemize}

\begin{itemize}
\tightlist
\item
  {Goshen/Elberta (}{39.983238, -111.990451,
  }{\href{https://www.google.com/url?q=https://www.google.com/maps/place/39\%25C2\%25B058'59.3\%2522N\%2B111\%25C2\%25B059'25.0\%2522W/@39.98315,-111.992484,706m/data\%3D!3m2!1e3!4b1!4m14!1m7!3m6!1s0x874d07d44e19369b:0xa13b930b4d04edb5!2sGoshen,\%2BUT\%2B84633!3b1!8m2!3d39.9530088!4d-111.9007716!3m5!1s0x0:0x0!7e2!8m2!3d39.9831504!4d-111.9902901\&sa=D\&ust=1564438983446000}{https://www.google.com/maps/place/39\%C2\%B058'59.3\%22N+111\%C2\%B059'25.0\%22W/@39.98315,-111.992484,706m/data=!3m2!1e3!4b1!4m14!1m7!3m6!1s0x874d07d44e19369b:0xa13b930b4d04edb5!2sGoshen,+UT+84633!3b1!8m2!3d39.9530088!4d-111.9007716!3m5!1s0x0:0x0!7e2!8m2!3d39.9831504!4d-111.9902901}}{)}
\end{itemize}

\begin{itemize}
\tightlist
\item
  {Further away (\textasciitilde{}40 min one-way)}
\item
  {No one else to worry about besides off-roaders and cows}
\item
  {No easy spot to land (rocky ground)}
\item
  {Lots of space to fly in, plenty to simulate the competition space}
\item
  {Our chosen spot to fly by the end of the year}
\end{itemize}

\begin{itemize}
\tightlist
\item
  {Most other areas that could be good around Provo are inside of the
  restricted airspace of an airport. Keep this in mind when looking for
  other spots to fly.'' - Tyler Critchfield}
\end{itemize}

\section{Airframe}

{``There is a good chance you will crash at some point. Choosing an
off-the-shelf airframe made rebuilds extremely easy, since parts could
simply be purchased and replaced. Even then, catastrophically crashed
airframes that seem irreparable can often be fixed with a little TLC in
the lab. During one flight test, we fell 100 feet straight down, broke
the airframe into 5ish pieces, and had it airworthy again in just a few
days. You just need to get really good with gorilla glue and water,
fiber tape (get the 3M packaging stuff- it will change your life!), and
making sure the hatch fits when you're done.'' -Ryan Anderson}

{}

{``The 3M tape that Ryan mentioned can be found here:
}{\href{https://www.google.com/url?q=https://www.amazon.com/Scotch-Shipping-Strapping-Designed-8959-RD/dp/B001AFKV0S/ref\%3Dsr_1_5?keywords\%3D3m\%2Bpacking\%2Btape\%2Bfiber\%2Btape\%26qid\%3D1562099039\%26s\%3Dgateway\%26sr\%3D8-5\&sa=D\&ust=1564438983448000}{https://www.amazon.com/Scotch-Shipping-Strapping-Designed-8959-RD/dp/B001AFKV0S/ref=sr\_1\_5?keywords=3m+packing+tape+fiber+tape\&qid=1562099039\&s=gateway\&sr=8-5}}{)''
- Tyler Critchfield}

{}

{``Airframes can take a rather large beating. Lots of tape and glue can
fix most things. A heat gun can fix the rest. It is a lot faster and
cheaper to repair than to make a new plane.'' -Brady Moon}

{}

{``The MFD Nimbus Pro is advertised as having a larger wingspan than the
MFD Crosswind. We bought both and they were identical.'' -Ryan Anderson}

{}

{``Before flying, make sure you do extensive range testing at each new
location. A quick test before each flight couldn't hurt either. A couple
of our crashes during the year were due to a loss of connection, which
if I remember right was a fault of the transmitter (something was
loose). We actually crashed a couple of times into a tree, which saved
the plane and would have been disastrous if it had crashed elsewhere.''
- Tyler Critchfield}

{}

{``Try experimenting with airspeed. See how it affects imaging and
waypoint accuracy in flight tests. Flying slower may result in higher
accuracy and image quality, but flying faster will help you finish the
mission faster. According to the 2019 rules, you have 20 minutes to
complete the competition for full points. If you decide to fly faster,
you may not need the foam tail wedges we added.'' -Ryan Anderson}

{}

{``Try and make a better camera mount that guarantees the camera points
straight down. Target geolocation was off and this could have
contributed to why.'' -Tyler Miller}

{}

{``Buy some new batteries'' - Jacob Willis}

{}

{``On the subject of batteries, make sure someone is able to give you a
safety training on their handling. If handled properly you shouldn't
have any problems, but it's always better to be cautious. One time I
unplugged the battery connection cord from the charger before
disconnecting the battery, and there was a huge shock and I'm lucky I
wasn't hurt. I think someone else on the team may have had a similar
experience. You probably want at least one person in charge of their
care, and ideally two or three people who know how to charge and
discharge them.'' - Tyler Critchfield}

{}

{``You may try testing a new airframe using R/C only (without all the
expensive electronics inside). Note that a lighter airframe can have
significantly different flight dynamics even if the center of gravity is
in the same location. This took our pilot by surprise once and caused a
serious crash.'' - Ryan Anderson}

\section{Controls}
\hypertarget{h.juthe769iomo}{\section{\texorpdfstring{{Controls}}{Controls}}\label{h.juthe769iomo}}

{``Look into pulling the latest ROSplane version from the MAGICC Lab
into the BYU-AUVSI ROSplane fork. Or maybe just use the latest MAGICC
Lab version.'' -Derek Knowles}

{}

{``ROSplane is good but not perfect. It might be because we were using
an older version but there are a number of edge cases that were not
covered by the code and led to undesirable behavior (e.g. when planning
fillet paths, if the angle is too small or large between the waypoints,
the halfplanes get set in incorrect locations). The vanilla ROSplane is
also set up to just fly straight line segments or orbit segments which
makes it a bit tricky to do a dedicated loiter'' - John Akagi}

{}

{``One of the biggest problems that I saw from both last year's code and
this years is that the airplane starts planning its path from where it
is but by the time it has finished the path, the plane has moved. This
can lead to the plane flying through areas it wasn't intending.'' - John
Akagi}

{}

{``Get a good simulation and simulation display up and running asap.
Gazebo works fine for simulating flight but is terrible for
understanding what the plane is actually doing. The
missionPlanner/plotter.py was my very last minute attempt to at least
get an idea of what the plane was doing.'' - John Akagi}

{}

{``Also might be a good idea to replace the Odroid on the plane or at
least have an extra. It's been through a few crashes'' - Derek Knowles}

{}

{``If you're not going to fly straight-line paths (straight-line path
manager), you have to make modifications to either the fillet/dubins
path managers themselves or be smarter about how you specify waypoints
in your path planner in order to avoid having your plane fly off the map
without warning.'' - Andrew Torgesen}

{}

{``Do extensive testing in simulation (make sure the simulation
works--it currently doesn't work), and use good programming practices
with git and unit testing as you add new functionality. It is worth the
time and effort to integrate good programming practices.'' - Andrew
Torgesen}

{}

{``Make sure you are not relying on one or two individuals to handle all
things related directly to autonomous flight. Most team members should
be proficient in ROS }{and should be able to debug ROSplane or other ROS
packages should issues arise}{, as they will. We were hit hard by the
fact that the controls subteam, consisting of three individuals, was
away for most of the summer leading up to the competition.'' - Andrew
Torgesen}

{}

{``Get up in the air as early in the development process as you can to
tune the ROSplane gains. That will make your life a lot easier and
facilitate testing for other subsystems that needs to get done.'' -
Andrew Torgesen}

{}

{``ROSPlane and most of the other MAGICC lab stuff have migrated to
Ubuntu 18.04 and ROS Melodic. It is probably a good idea to follow, but
make sure you know how the whole system works and flies before
switching'' - Jacob Willis}

\section{Imaging}

{}

{``I mainly worked on the image classification gui in the
BYU-AUVSI/imaging repository. Suggested improvements can be found in the
header of the main gui.py and inside of each tab file inside of
imaging/client/lib. Tkinter is pretty basic and there a number of random
bugs with the gui. When you have questions about design rationale,
comments in the code, etc., please reach out to me'' - Derek Knowles}

{}

{``I built the server and some of the autonomous imaging stuff. A few
suggested improvements are listed in github issues on the Imaging repo.
The server is pretty stable and the items listed are enhancements.
Please reach out to me and I can meet with you in person to explain how
to use/improve it.'' -Tyler Miller}

{}

{``We never had a chance to extensively test target geolocation in
flight. It would be worth testing and tuning. Day before competition
target geolocation was off by 250+ft'' -Tyler Miller}

{}

{``Even though we were the only team to score an autonomous ODLC,
there's still a lot of work to be done on the autonomous system. One of
our main holdups was having a good, large library of image data to test
against. There's a hard drive in our cupboard with a bunch of data that
we either collected or generated this past year. It will require some
organization/ categorization (ie: labeling images that have targets in
them, etc), but it will be worth it if your looking to improve
autonomous.'' - Tyler Miller}

{}

{``Again with autonomous, here's what we wanted to improve, but didn't
have time: We wanted to add a class to our letter classifier for
`no-letter' (similar to the shape classifier's `notarget'), and then use
this to help reject false-positives. We also wanted to make a net for
color classification, as we had a hard time getting this to work
accurately (whether a simple net or a CNN is best for this, we're not
sure). Our strategy for orientation classification worked okay but not
great, and it would be good to brainstorm ways it could improve. Jake
Johnson was the main lead on autonomous development (he's in the MAGICC
lab) and I helped with creating the datasets and nets.'' - Tyler Miller}

{}

{``Find someway to access a computer with an NVidia card and CUDA
installed on linux. It will save you so much time when training the
nets'' -Tyler Miller}

\section{UGV}

{``Progress on the UGV was slow. We made a good choice of parachute and
a fairly reliable payload bay, but we didn't have a good system for
testing the UGV drop code until it was too late to make improvements to
it. Having a plane that you can fly an autopilot on and having code that
is modular enough to not get stuck waiting for other people to finish
will help you make better use of your flight test time.'' - Jacob
Willis}

{}

{``The UGV electronics consisted of a battery, a 5V BEC, a Openpilot
Revo board running custom code based on Airbourne\_f4, a gps, and a 433
MHz transmitter. Turns out the transmitter is not FCC compliant, so it
can't be used in the competition - I'd recommend getting a set of the
3DR 915 MHz radios for communicating. The flight tents are only
\textasciitilde{}200 yards from where the UGV drop was, so range
shouldn't be a huge concern. Getting a good heading estimate was hard -
GPS was too random and the magnetometer wasn't reliable. I would
recommend building a car that has encoders on it so you can at leas
drive a straight line far enough to get a good heading estimate.'' -
Jacob Willis}

{}

{``The runway where the UGV drop was had lots of weeds growing in the
cracks. It looked like they mowed it beforehand, but even then our
little UGV would not have made it through the weeds.'' - Jacob Willis}

{}

{``The most successful UGVs had 4'' tires on a platform that could land
on either side. Only 3 made it to the drive target - this will likely be
a major part of the competition next year as well.'' - Jacob Willis}

\section{Competition}

{``Be fully prepared to fly when you arrive at the competition. You have
minimal setup time and it's crucial that you be in the air as soon as
possible once it's your turn to fly.'' - Derek Knowles}

{}

{``During setup, do not get casual or overly rushed with the checklist.
Even though you have limited setup time and it may be tempting to take
shortcuts, you may regret it in the end. We forgot to update two
parameters on the flight line and, as a result, our telemetry was off by
about 50 miles. This would have been avoided had we stuck to the
checklist.'' -Ryan Anderson}

{}

{``Your time in Maryland will be much more enjoyable, less stressful,
and more fulfilling if you check an UAS at the airport that is
competition-ready. You'll have some time minor improvements, but don't
expect to have time to fix major problems while you're there.'' -Ryan
Anderson}

{}

{``It seems that most teams spend a lot of time on webster field
watching and supporting other teams. I think doing this does a lot to
build a relationship between teams and universities and I wish that we
had been more prepared so we could have spent some time making
connections.}

{The judges did notice that we weren't around for the first two days of
the competition.'' - Jacob Willis}

{}

{``Things got very rushed after we got out to the flight line. Roles had
been defined, but not well practiced. The tent we were given had a table
with three chairs. This was just enough to set up the groundstation, but
we needed much more space for the imaging team. We drove our van out to
the flight line, so we set up the imaging team in the van. This worked
very well.'' - Jacob Willis}

{}

{``-To avoid loss of power, we used the power from the van to run our
network. This worked fine, though it would have been nice (and more
professional) to have a backup supply. We used the competition power to
keep laptops charged. There were no issues with the power (that I know
of).}

{-We got the interop cable as soon as we arrived, so there was lots of
time to get mission data, and plan paths.}

{- Our pilot was apprised of the waypoints we were flying, but didn't
know where the search area was. This reduced the ability we had to find
targets.}

{- Targets that were imaged were very easy to manually classify. The
camera gave very clear pictures, even from our high altitude.}

{- The UGV drop is extremely close to the viewer stands, which makes it
very important to have good control of the aircraft while doing this. We
received an infraction for crossing that boundary, losing all of our
autonomous flight points. In addition we had a low altitude infraction
while attempting autonomous UGV drop because we set our altitude to 30m
instead of the typical 40m we were using for the other waypoints. }

{- Because we weren't confident in our path manager, we loitered for
three minutes before attempting to follow the waypoint path. This cut
into our imaging time significantly.}

{- It is wise to keep the camera off unless over the search area. This
both helps to reduce the number of empty images that must be manually
processed and it reduces the liklihood of a false positive from the
autonomous imaging system. To aid this, consider modifying and
distributing the missionPlotter in Metis so anyone on the team can have
a live update of where the plane is.}

{- The waypoint path might be slightly too long for keeping a telemetry
connection. I think the judges were not impressed by the spotty
telemetry we got at the farthest end of the path. It would be good to
discuss this with Craig (MAGICC lab) and find out if there is a good way
to do a combination of a close range omni antenna and a long range,
high-gain antenna. This would require a different lightbeam antenna.
Consider getting the Ubiquiti Rocke Prism AC along with one of
Ubiquiti's long range antennas.}

{- It was very windy while we were there. This added significant
disturbance to our flight and made the path follower and altitude hold
controllers work hard to stay on track. Both of these could use more
tuning.}

{- Make sure to be keeping track of the mission time. We used
\textasciitilde{}28 minutes. The breakdown was approximately:}

{~ - 1 minute - Takeoff}

{~ - 3 minutes - Autonomous loitering (to make sure we had Autonomous
flight)}

{~ - 11 minutes - flying waypoints}

{~ - 5 minutes - attempting UGV drop}

{~ - 5-6 minutes - search path}

{~ - 1-2 minutes - wrapping up target submission and landing.}

{Since our imaging system was so good and our UGV so unreliable, it
would have been much better for us to fly more of the search area than
to even attempt the UGV drop.}

{There was a decision to perform manual search because the autonomous
search was too long. Much of the area we searched was searched before -
having a smarter way of planning the autonomous search, or having it
plan wider paths that get \textasciitilde{}70\% of the search area fast
might be a better approach.}

{- The judges mentioned they want to see more preflight testing being
done by all of the teams.}

{- Many teams had their groundstation in a pelican case. We purchased a
plastic tub for it and found it extremely helpful to keep things
organized and plugged in. It is ready to go for next year.'' - Jacob
Willis}

\section{Travel}

{``The airframe box was oversized and cost upwards of \$230 each way to
check on the airline. You might look into FedEx overnighting it.'' -
Ryan Anderson}

{}

{``Last year's team had difficulty finding a place to perform flight
tests in Maryland before their competition time. Chaptico Park was
relatively close, had plenty of space to fly, and was fairly deserted
throughout our stay. There was a power outlet and a covered area with
tables that was very convenient for setup.'' - Ryan Anderson}

{}

{``If you plan to go to chaptico park bring 150' of extension cords'' -
Jacob Willis}

{}

{``We transported our equipment in one large cardboard box and three
suitcases. The large box had two fuselages and wings in it, while the
suitcases had tools, electronics and other equipment. The box was
oversize, so it was \$230 each way to fly it.'' - Jacob Willis}

{``BWI is the closest airport to the competition, but we all wanted to
see Washington DC, so it might be worthwhile to try to fly into Dulles
or Reagan.'' - Jacob Willis}

{``Capstone didn't pay for any of the rental cars or hotel rooms before
we arrived, so we had to put them on our credit cards. Make sure there
are people on the team with high enough credit limits to do this (or,
even better, make sure capstone pays for things before you leave).'' -
Jacob Willis}

\section{Other Subjects}

{``Make a solid effort on the paper and videos - they will take longer
to make than you think they will. Luckily we had Brandon who knew how to
put a video together, but otherwise we would've had a poor video to
submit. This could be a good assignment for an undergrad who wants to
help - they could be in charge of filming tests, taking pictures of
candid work sessions, etc. Even set up an instagram account to build
public interest. The video and paper will be much easier if you plan for
them in advance (for instance, make sure everyone reads the requirements
in September when the rules come out and review them periodically.'' -
Tyler Critchfield}

{}

{``Be consistent with taking pictures and video as you develop the
system so you have plenty of material to work with for the videos.'' -
John Akagi}

{}

{``I think most of our success in the paper and video came from trying
to model them after the highest-scoring papers and videos from the
previous year. Pay attention to and follow exactly what the judges are
looking for. Typically they want to see data - plots and tables - that
show you've tested the system and have an idea of how it will perform.''
- Tyler Critchfield}

{}

{``To save some time, I would consider asking Capstone if your final
report for the class can instead be the competition paper. Then you
don't need to write up multiple reports when everything is in crunch
time.'' - Tyler Critchfield}

\begin{center}\rule{0.5\linewidth}{\linethickness}\end{center}



\section{Appendix A: Contact Information}

\begin{longtable}[]{@{}lll@{}}
\toprule
\begin{minipage}[t]{0.30\columnwidth}\raggedright\strut
{Name}\strut
\end{minipage} & \begin{minipage}[t]{0.30\columnwidth}\raggedright\strut
{Email}\strut
\end{minipage} & \begin{minipage}[t]{0.30\columnwidth}\raggedright\strut
{Phone}\strut
\end{minipage}\tabularnewline
\begin{minipage}[t]{0.30\columnwidth}\raggedright\strut
{Akagi, John}\strut
\end{minipage} & \begin{minipage}[t]{0.30\columnwidth}\raggedright\strut
{akagi94@gmail.com}\strut
\end{minipage} & \begin{minipage}[t]{0.30\columnwidth}\raggedright\strut
{858-231-4416}\strut
\end{minipage}\tabularnewline
\begin{minipage}[t]{0.30\columnwidth}\raggedright\strut
{Anderson, Ryan}\strut
\end{minipage} & \begin{minipage}[t]{0.30\columnwidth}\raggedright\strut
{rymanderson@gmail.com}\strut
\end{minipage} & \begin{minipage}[t]{0.30\columnwidth}\raggedright\strut
{208-789-4318}\strut
\end{minipage}\tabularnewline
\begin{minipage}[t]{0.30\columnwidth}\raggedright\strut
{Critchfield, Tyler}\strut
\end{minipage} & \begin{minipage}[t]{0.30\columnwidth}\raggedright\strut
{trcritchfield@gmail.com}\strut
\end{minipage} & \begin{minipage}[t]{0.30\columnwidth}\raggedright\strut
{206-939-8274}\strut
\end{minipage}\tabularnewline
\begin{minipage}[t]{0.30\columnwidth}\raggedright\strut
{Eves, Kameron}\strut
\end{minipage} & \begin{minipage}[t]{0.30\columnwidth}\raggedright\strut
{ccackam@gmail.com}\strut
\end{minipage} & \begin{minipage}[t]{0.30\columnwidth}\raggedright\strut
{702-686-210}\strut
\end{minipage}\tabularnewline
\begin{minipage}[t]{0.30\columnwidth}\raggedright\strut
{Johnson, Jake}\strut
\end{minipage} & \begin{minipage}[t]{0.30\columnwidth}\raggedright\strut
{jacobcjohnson13@gmail.com}\strut
\end{minipage} & \begin{minipage}[t]{0.30\columnwidth}\raggedright\strut
{801-664-7586}\strut
\end{minipage}\tabularnewline
\begin{minipage}[t]{0.30\columnwidth}\raggedright\strut
{Knowles, Derek}\strut
\end{minipage} & \begin{minipage}[t]{0.30\columnwidth}\raggedright\strut
{knowles.derek@gmail.com}\strut
\end{minipage} & \begin{minipage}[t]{0.30\columnwidth}\raggedright\strut
{405-471-4285}\strut
\end{minipage}\tabularnewline
\begin{minipage}[t]{0.30\columnwidth}\raggedright\strut
{McBride, Brandon}\strut
\end{minipage} & \begin{minipage}[t]{0.30\columnwidth}\raggedright\strut
{brandon.mcbride4@gmail.com}\strut
\end{minipage} & \begin{minipage}[t]{0.30\columnwidth}\raggedright\strut
{801-520-9165}\strut
\end{minipage}\tabularnewline
\begin{minipage}[t]{0.30\columnwidth}\raggedright\strut
{Miller, Tyler}\strut
\end{minipage} & \begin{minipage}[t]{0.30\columnwidth}\raggedright\strut
{tylerm15@gmail.com}\strut
\end{minipage} & \begin{minipage}[t]{0.30\columnwidth}\raggedright\strut
{385-399-3472}\strut
\end{minipage}\tabularnewline
\begin{minipage}[t]{0.30\columnwidth}\raggedright\strut
{Moon, Brady}\strut
\end{minipage} & \begin{minipage}[t]{0.30\columnwidth}\raggedright\strut
{bradygmoon@gmail.com}\strut
\end{minipage} & \begin{minipage}[t]{0.30\columnwidth}\raggedright\strut
{435-828-5858}\strut
\end{minipage}\tabularnewline
\begin{minipage}[t]{0.30\columnwidth}\raggedright\strut
{Olsen, Connor}\strut
\end{minipage} & \begin{minipage}[t]{0.30\columnwidth}\raggedright\strut
{connorolsen72@gmail.com}\strut
\end{minipage} & \begin{minipage}[t]{0.30\columnwidth}\raggedright\strut
{385-230-3932}\strut
\end{minipage}\tabularnewline
\begin{minipage}[t]{0.30\columnwidth}\raggedright\strut
{Torgesen, Andrew}\strut
\end{minipage} & \begin{minipage}[t]{0.30\columnwidth}\raggedright\strut
{andrew.torgesen@gmail.com}\strut
\end{minipage} & \begin{minipage}[t]{0.30\columnwidth}\raggedright\strut
{661-210-5214}\strut
\end{minipage}\tabularnewline
\begin{minipage}[t]{0.30\columnwidth}\raggedright\strut
{Willis, Jacob}\strut
\end{minipage} & \begin{minipage}[t]{0.30\columnwidth}\raggedright\strut
{jbwillis272@gmail.com}\strut
\end{minipage} & \begin{minipage}[t]{0.30\columnwidth}\raggedright\strut
{208-206-1780}\strut
\end{minipage}\tabularnewline
\bottomrule
\end{longtable}

{}

\begin{center}\rule{0.5\linewidth}{\linethickness}\end{center}

\section{Appendix B: Help Index by Person}

{We are all committed to helping you out. Please reach out to one or
multiple people when you have questions or issues.}

{}


\begin{longtable}[]{@{}lll@{}}
\toprule
\begin{minipage}[t]{0.30\columnwidth}\raggedright\strut
{Name}\strut
\end{minipage} & \begin{minipage}[t]{0.30\columnwidth}\raggedright\strut
{Subteam(s)}\strut
\end{minipage} & \begin{minipage}[t]{0.30\columnwidth}\raggedright\strut
{Specialties}\strut
\end{minipage}\tabularnewline
\begin{minipage}[t]{0.30\columnwidth}\raggedright\strut
{Akagi, John}\strut
\end{minipage} & \begin{minipage}[t]{0.30\columnwidth}\raggedright\strut
{Controls}\strut
\end{minipage} & \begin{minipage}[t]{0.30\columnwidth}\raggedright\strut
{Metis, Theseus,}\strut
\end{minipage}\tabularnewline
\begin{minipage}[t]{0.30\columnwidth}\raggedright\strut
{Anderson, Ryan}\strut
\end{minipage} & \begin{minipage}[t]{0.30\columnwidth}\raggedright\strut
{Airframe, Controls}\strut
\end{minipage} & \begin{minipage}[t]{0.30\columnwidth}\raggedright\strut
{Aircraft aerodynamics, weight distribution, Groundstation, plane
construction}\strut
\end{minipage}\tabularnewline
\begin{minipage}[t]{0.30\columnwidth}\raggedright\strut
{Critchfield, Tyler}\strut
\end{minipage} & \begin{minipage}[t]{0.30\columnwidth}\raggedright\strut
{Airframe}\strut
\end{minipage} & \begin{minipage}[t]{0.30\columnwidth}\raggedright\strut
{Aircraft aerodynamics}\strut
\end{minipage}\tabularnewline
\begin{minipage}[t]{0.30\columnwidth}\raggedright\strut
{Eves, Kameron}\strut
\end{minipage} & \begin{minipage}[t]{0.30\columnwidth}\raggedright\strut
{Airframe}\strut
\end{minipage} & \begin{minipage}[t]{0.30\columnwidth}\raggedright\strut
{Safety Pilot}\strut
\end{minipage}\tabularnewline
\begin{minipage}[t]{0.30\columnwidth}\raggedright\strut
{Johnson, Jake}\strut
\end{minipage} & \begin{minipage}[t]{0.30\columnwidth}\raggedright\strut
{Vision}\strut
\end{minipage} & \begin{minipage}[t]{0.30\columnwidth}\raggedright\strut
{Autonomous target recognition}\strut
\end{minipage}\tabularnewline
\begin{minipage}[t]{0.30\columnwidth}\raggedright\strut
{Knowles, Derek}\strut
\end{minipage} & \begin{minipage}[t]{0.30\columnwidth}\raggedright\strut
{Vision, UGV}\strut
\end{minipage} & \begin{minipage}[t]{0.30\columnwidth}\raggedright\strut
{Image classification GUI, Payload planner \& release}\strut
\end{minipage}\tabularnewline
\begin{minipage}[t]{0.30\columnwidth}\raggedright\strut
{McBride, Brandon}\strut
\end{minipage} & \begin{minipage}[t]{0.30\columnwidth}\raggedright\strut
{Vision, UGV}\strut
\end{minipage} & \begin{minipage}[t]{0.30\columnwidth}\raggedright\strut
{UGV drive, imaging server client}\strut
\end{minipage}\tabularnewline
\begin{minipage}[t]{0.30\columnwidth}\raggedright\strut
{Miller, Tyler}\strut
\end{minipage} & \begin{minipage}[t]{0.30\columnwidth}\raggedright\strut
{Vision}\strut
\end{minipage} & \begin{minipage}[t]{0.30\columnwidth}\raggedright\strut
{Interoperability, autonomous target recognition, imaging server}\strut
\end{minipage}\tabularnewline
\begin{minipage}[t]{0.30\columnwidth}\raggedright\strut
{Moon, Brady}\strut
\end{minipage} & \begin{minipage}[t]{0.30\columnwidth}\raggedright\strut
{Controls}\strut
\end{minipage} & \begin{minipage}[t]{0.30\columnwidth}\raggedright\strut
{Metis, Path planning, RRT}\strut
\end{minipage}\tabularnewline
\begin{minipage}[t]{0.30\columnwidth}\raggedright\strut
{Olsen, Connor}\strut
\end{minipage} & \begin{minipage}[t]{0.30\columnwidth}\raggedright\strut
{Vision, UGV}\strut
\end{minipage} & \begin{minipage}[t]{0.30\columnwidth}\raggedright\strut
{Geolocation, UGV drive}\strut
\end{minipage}\tabularnewline
\begin{minipage}[t]{0.30\columnwidth}\raggedright\strut
{Torgesen, Andrew}\strut
\end{minipage} & \begin{minipage}[t]{0.30\columnwidth}\raggedright\strut
{Controls}\strut
\end{minipage} & \begin{minipage}[t]{0.30\columnwidth}\raggedright\strut
{Groundstation, ROSplane, all things ROS, LaTeX, High-level software
architecture}\strut
\end{minipage}\tabularnewline
\begin{minipage}[t]{0.30\columnwidth}\raggedright\strut
{Willis, Jacob}\strut
\end{minipage} & \begin{minipage}[t]{0.30\columnwidth}\raggedright\strut
{UGV, Controls}\strut
\end{minipage} & \begin{minipage}[t]{0.30\columnwidth}\raggedright\strut
{UGV drop \& drive, Avionics, Metis, ROSPlane}\strut
\end{minipage}\tabularnewline
\bottomrule
\end{longtable}

{}

\begin{center}\rule{0.5\linewidth}{\linethickness}\end{center}


\section{Appendix C: Help Index by Resource}

{This is meant to list the various aspects of the AUVSI competition and
provide you with a contact for each subject matter. See Appendix A for
contact information for each individual. }

\hypertarget{h.fzoo3toav61z}{\subsection{\texorpdfstring{{BYU-AUVSI
Github
Repositories:}}{BYU-AUVSI Github Repositories:}}\label{h.fzoo3toav61z}}

\hypertarget{h.ez5i7wokvzkz}{\subsubsection{\texorpdfstring{{A6000\_ros:
Tyler Miller}}{A6000\_ros: Tyler Miller}}\label{h.ez5i7wokvzkz}}

{(tools for using a6000 in ROs)}

\hypertarget{h.40v6z6uc89r7}{\subsubsection{\texorpdfstring{{Airbourne\_f4:
Jacob Willis}}{Airbourne\_f4: Jacob Willis}}\label{h.40v6z6uc89r7}}

{This is a fork of the board support package used by rosflight. We used
it for the UGV. Most of the code we wrote for the UGV is contained
here.}

\hypertarget{h.smcq1h7rkxh}{\subsubsection{\texorpdfstring{{auvsi\_documentation\_2019:
Andrew
Torgesen}}{auvsi\_documentation\_2019: Andrew Torgesen}}\label{h.smcq1h7rkxh}}

{(Source code for all reports and documentation)}

\hypertarget{h.mabru5lpfx9n}{\subsubsection{\texorpdfstring{{Imaging:
Tyler Miller}}{Imaging: Tyler Miller}}\label{h.mabru5lpfx9n}}

{Imaging/autonomous: Jake Johnson}

{Imaging/client: Derek Knowles}

{Imaging/server: Tyler Miller~~~~~~~~}

\hypertarget{h.n49nu7bx5vod}{\subsubsection{\texorpdfstring{{Interop\_pkg:
Tyler Miller}}{Interop\_pkg: Tyler Miller}}\label{h.n49nu7bx5vod}}

{(Interoperability stuff as a ros package)}

\hypertarget{h.3xc7veylyan1}{\subsubsection{\texorpdfstring{{Inertial\_sense\_ros:
Andrew
Torgesen}}{Inertial\_sense\_ros: Andrew Torgesen}}\label{h.3xc7veylyan1}}

{(ROS wrapper for the INertialSense uINS2 GPS-INS sensor}

\hypertarget{h.md8utgthwf1}{\subsubsection{\texorpdfstring{{Metis: John
Akagi}}{Metis: John Akagi}}\label{h.md8utgthwf1}}

{(Mission and path planner)}

{~~~~~~~~Metis/pathPlanner: Brady Moon}

{~~~~~~~~Metis/missionPlanner/payloadDrop.py: Derek Knowles}

{~~~~~~~~Metis/missionPlanner/payloadPlanner.py: Derek Knowles}

\hypertarget{h.oze4xaqxxlu9}{\subsubsection{\texorpdfstring{{ros\_groundstation
(AUVSI-SUAS-2019 branch) : Andrew
Torgesen}}{ros\_groundstation (AUVSI-SUAS-2019 branch) : Andrew Torgesen}}\label{h.oze4xaqxxlu9}}

{(RQT plugin for interacting with ROSplane)}

\hypertarget{h.phcluy8fl6f3}{\subsubsection{\texorpdfstring{{Rosflight:
James Jackson
(}{\href{mailto:superjax08@gmail.com}{\nolinkurl{superjax08@gmail.com}}}{)}}{Rosflight: James Jackson (superjax08@gmail.com)}}\label{h.phcluy8fl6f3}}

{(ROS stack for the ROSflight autopilot)}

{}

\hypertarget{h.rfp5kydue3f1}{\subsubsection{\texorpdfstring{{Rosplane:
James Jackson
(}{superjax08@gmail.com}{)}}{Rosplane: James Jackson (superjax08@gmail.com)}}\label{h.rfp5kydue3f1}}

{~~~~~~~~(fixed-wing autopilot for ROS)}

{~~~~~~~~We used the ``plane'' branch on the odroid}

{}

\hypertarget{h.hbcl1x4i3t9h}{\subsubsection{\texorpdfstring{{Scripts:
Tyler Miller}}{Scripts: Tyler Miller}}\label{h.hbcl1x4i3t9h}}

{(scripts for ground station and onboard computers)}

\hypertarget{h.6tkhkdxr0376}{\subsubsection{\texorpdfstring{{TechnicalDesignPaper:
Andrew
Torgesen}}{TechnicalDesignPaper: Andrew Torgesen}}\label{h.6tkhkdxr0376}}

{(Technical Design paper for the AUVSI copetition)}

\hypertarget{h.gpyymqyjiqy0}{\subsubsection{\texorpdfstring{{Uav\_msgs:
John Akagi}}{Uav\_msgs: John Akagi}}\label{h.gpyymqyjiqy0}}

{(shared messages between ground station, interop, imaging, and
autopilot)}

\hypertarget{h.lnuk2wx0zogg}{\subsubsection{\texorpdfstring{{Ugv\_auto\_drive:
Jacob Willis}}{Ugv\_auto\_drive: Jacob Willis}}\label{h.lnuk2wx0zogg}}

{This was meant to be a higher level program that used Airbourne\_f4. We
ended up just developing in the Airbourne\_f4 repo (not the best choice
- it was just easier). There's not much here, mostly some random tidbits
of code.}

\hypertarget{h.d645jwbpz1js}{\subsection{\texorpdfstring{{Capstone:}}{Capstone:}}\label{h.d645jwbpz1js}}

\hypertarget{h.p8pvzhwgals7}{\subsubsection{\texorpdfstring{{Documentation:
Andrew
Torgesen}}{Documentation: Andrew Torgesen}}\label{h.p8pvzhwgals7}}

\hypertarget{h.4u5ohftr3ivt}{\subsubsection{\texorpdfstring{{Purchasing:
Jacob Willis}}{Purchasing: Jacob Willis}}\label{h.4u5ohftr3ivt}}

\hypertarget{h.hdq7osdifj4b}{\subsection{\texorpdfstring{{Design \&
Development:}}{Design \& Development:}}\label{h.hdq7osdifj4b}}

\hypertarget{h.ksxlyips5td6}{\subsubsection{\texorpdfstring{{Airframe:
Ryan Anderson \& Tyler
Critchfield}}{Airframe: Ryan Anderson \& Tyler Critchfield}}\label{h.ksxlyips5td6}}

\hypertarget{h.uewzmm294rc5}{\subsection{\texorpdfstring{{Competition
Roles:}}{Competition Roles:}}\label{h.uewzmm294rc5}}

\hypertarget{h.g1aru5oodxem}{\subsubsection{\texorpdfstring{{Team
Captain: Andrew
Torgesen}}{Team Captain: Andrew Torgesen}}\label{h.g1aru5oodxem}}

\hypertarget{h.v9u227xrzjux}{\subsubsection{\texorpdfstring{{Ground
Station: Andrew
Torgesen}}{Ground Station: Andrew Torgesen}}\label{h.v9u227xrzjux}}

\hypertarget{h.5qjlfzs1vco2}{\subsubsection{\texorpdfstring{{Safety
Pilot: Kameron
Eves}}{Safety Pilot: Kameron Eves}}\label{h.5qjlfzs1vco2}}

\hypertarget{h.onmum1m8p9el}{\subsubsection{\texorpdfstring{{Antenna
Pointing: Connor
Olsen}}{Antenna Pointing: Connor Olsen}}\label{h.onmum1m8p9el}}

\hypertarget{h.9vddyurbqkh1}{\subsubsection{\texorpdfstring{{Past
Competitions Overview: Dr. Tim
McLain}}{Past Competitions Overview: Dr. Tim McLain}}\label{h.9vddyurbqkh1}}

\begin{center}\rule{0.5\linewidth}{\linethickness}\end{center}

\section{\texorpdfstring{{}}{}}\label{h.r61o3pbzck6b}

\hypertarget{h.ejo1f4jddl42}{\section{\texorpdfstring{{Appendix D:
Passwords and IP
Addresses}}{Appendix D: Passwords and IP Addresses}}\label{h.ejo1f4jddl42}}


\begin{longtable}[]{@{}lll@{}}
\toprule
\begin{minipage}[t]{0.30\columnwidth}\raggedright\strut
{Item}\strut
\end{minipage} & \begin{minipage}[t]{0.30\columnwidth}\raggedright\strut
{Password}\strut
\end{minipage} & \begin{minipage}[t]{0.30\columnwidth}\raggedright\strut
{IP Address}\strut
\end{minipage}\tabularnewline
\begin{minipage}[t]{0.30\columnwidth}\raggedright\strut
{Cupboard Bicycle Lock}\strut
\end{minipage} & \begin{minipage}[t]{0.30\columnwidth}\raggedright\strut
{1413}\strut
\end{minipage} & \begin{minipage}[t]{0.30\columnwidth}\raggedright\strut
{}\strut
\end{minipage}\tabularnewline
\begin{minipage}[t]{0.30\columnwidth}\raggedright\strut
{Odroid}\strut
\end{minipage} & \begin{minipage}[t]{0.30\columnwidth}\raggedright\strut
{odroid}\strut
\end{minipage} & \begin{minipage}[t]{0.30\columnwidth}\raggedright\strut
{192.168.1.8}\strut
\end{minipage}\tabularnewline
\begin{minipage}[t]{0.30\columnwidth}\raggedright\strut
{Brix}\strut
\end{minipage} & \begin{minipage}[t]{0.30\columnwidth}\raggedright\strut
{Byuauvs1}\strut
\end{minipage} & \begin{minipage}[t]{0.30\columnwidth}\raggedright\strut
{192.168.1.10}\strut
\end{minipage}\tabularnewline
\begin{minipage}[t]{0.30\columnwidth}\raggedright\strut
{Imaging Server}\strut
\end{minipage} & \begin{minipage}[t]{0.30\columnwidth}\raggedright\strut
{}\strut
\end{minipage} & \begin{minipage}[t]{0.30\columnwidth}\raggedright\strut
{192.168.1.10:5000}\strut
\end{minipage}\tabularnewline
\begin{minipage}[t]{0.30\columnwidth}\raggedright\strut
{Router}\strut
\end{minipage} & \begin{minipage}[t]{0.30\columnwidth}\raggedright\strut
{byuauvsi}\strut
\end{minipage} & \begin{minipage}[t]{0.30\columnwidth}\raggedright\strut
{192.168.1.1}\strut
\end{minipage}\tabularnewline
\begin{minipage}[t]{0.30\columnwidth}\raggedright\strut
{Lightbeam}\strut
\end{minipage} & \begin{minipage}[t]{0.30\columnwidth}\raggedright\strut
{}\strut
\end{minipage} & \begin{minipage}[t]{0.30\columnwidth}\raggedright\strut
{}\strut
\end{minipage}\tabularnewline
\begin{minipage}[t]{0.30\columnwidth}\raggedright\strut
{Bullet}\strut
\end{minipage} & \begin{minipage}[t]{0.30\columnwidth}\raggedright\strut
{}\strut
\end{minipage} & \begin{minipage}[t]{0.30\columnwidth}\raggedright\strut
{}\strut
\end{minipage}\tabularnewline
\begin{minipage}[t]{0.30\columnwidth}\raggedright\strut
{}\strut
\end{minipage} & \begin{minipage}[t]{0.30\columnwidth}\raggedright\strut
{}\strut
\end{minipage} & \begin{minipage}[t]{0.30\columnwidth}\raggedright\strut
{}\strut
\end{minipage}\tabularnewline
\bottomrule
\end{longtable}

{}

{}

{}

{}

\end{document}
