\documentclass[]{auvsi_doc}
\setkeys{auvsi_doc.cls}{
	AUVSITitle={Unmanned Ground Vehicle Drop Subsystem Summary},
	AUVSILogoPath={./figs/logo.pdf},
}

% include extra packages, if needed

% Remove Heading Numbers
\setcounter{secnumdepth}{0}

% Remove Heading Numbers
\setcounter{secnumdepth}{0}


% include extra packages, if needed

\begin{document}

\begin{AUVSITitlePage}
\begin{artifacttable}
\entry{GV-005, 1.0, 10-30-2018, Wrote concept description, Kameron Eves, Andrew Torgesen}
\entry{GV-005, 1.1, 2-21-2019, Updated to reflect subsystem engineering, Jacob Willis, Brandon McBride}
\entry{GV-005, 1.2, 3-01-2019, Added performance summary and remaining development, Jacob Willis, Derek Knowles}
\entry{GV-005, 1.2, 4-05-2019, Summarized results, Jacob Willis, Derek Knowles}
% additional \entry{} commands for extra rows in the revision table, if needed
\end{artifacttable}
\end{AUVSITitlePage}

% document contents (see below for LaTex commands that make your life easier)
\section{Introduction}
This document gives a more detailed description of the selected concept for the UGV delivery system. The concept selected during concept development was a parachute with fins. After considering the added complexity of fins, and the small benefit they provide, we determined to simply use a parachute.

\section{Description}

The UGV is to be loaded within the aircraft.
Upon a command from the flight controller system, a small hatch opens and the UGV falls out. 
The UGV is carried to the ground by a lightweight 36 inch nylon parachute, purchased from FruityChutes. 
The parachute is loaded onto the aircraft in a tube that allows the UGV to pull it out of the aircraft as it falls. This helps stop the tangling that can come from a folded parachute. To also prevent tangling, and to make for a more predictable drop, the parachute is folding according to GV-007. After exiting the aircraft the parachute will be opened by drag. This will slow down the system enough to allow the UGV to survive impact without damage. A visual depiction of our chosen system can be seen in Fig.~\ref{fig:side}.

\begin{figure}[h]
\centering
\includegraphics[width=90mm]{./figs/Parachute_Side.jpg}
\caption{Our parachute and simulated UGV as seen from the side.}
\label{fig:side}
\end{figure}

An accurate landing is an important part of the competition, and is the key success measure governing the design of the UGV drop system.
A hole in the top of the parachute improves the horizontal accuracy of the system by allowing a faster drop, but with more consistent aerodynamic effects.
This hole is known in the industry as a spill hole because it allows the air to spill out of the center of the parachute. 
 While this will not be enough to correct for large errors, it should be enough to ensure the system doesn't drift  too randomly. 
 
 The algorithm for dropping objects from a UGV, as detailed in \textit{Small Unmanned Aircraft: Theory and Practice} by Randy Beard and Tim McLain, is used to estimate the proper location to drop the UGV from in order to hit the target. This algorithm uses the wind and velocity of the aircraft to predict the best location to release the payload. A simulation of the drop, based on this algorithm is in GV-008. 
 The parachute adds some complexity to the algorithm, since the surface area and coefficient of drag are not constant during the drop. In order to more accurately predict the landing location of the UGV, the drop will be repeatedly timed, and the time will be used by the algorithm. This will allow us better accuracy in our drop algorithm than using a model of how the parachute drops. It also necessitates a highly repeatable deployment. These were design goals, and were focused on in the development of GV-006 and GV-007.

\section{Performance Summary}
The UGV subsystem performs within our defined acceptable range for all performance measures.
Due to its simple design, the mechanism weighs less and supports much more weight than expected. 
The parachute we selected is compact, yet allows a maximum landing velocity of three meters per second. This is sufficient to ensure a gentle landing of the UGV, meeting the competition requirements for the UGV drop. Drop accuracy is within our acceptable range and will earn us points in the competition.

\section{Conclusion}
Using the system described above, we have tested the drop accuracy of the UGV to be within 65 feet. This number will improve as the full system is continued to be refined before the competition. 
This is considered fair in our key success measures and will earn us 25\% of the points possible for this portion of the competition.
\end{document}

