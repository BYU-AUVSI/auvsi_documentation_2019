\documentclass[]{auvsi_doc}
\setkeys{auvsi_doc.cls}{
	AUVSITitle={Unmanned Ground Vehicle Payload Drop Planner Description},
	AUVSILogoPath={./figs/logo.pdf},
}

% include extra packages, if needed

% Remove Heading Numbers
\setcounter{secnumdepth}{0}

% Remove Heading Numbers
\setcounter{secnumdepth}{0}


% include extra packages, if needed

\begin{document}

\begin{AUVSITitlePage}
\begin{artifacttable}
\entry{GV-009, 1.0, 04-05-2019, Initial creation, Derek Knowles, Jacob Willis}
% additional \entry{} commands for extra rows in the revision table, if needed
\end{artifacttable}
\end{AUVSITitlePage}


\section{Indroduction}
This document describes the design of our payload planner software. The payload planner calculates the location to drop the UGV from, given a landing location, obstacle locations, and estimates of the UAS and wind states.
The path planned for a simulated environment including two obstacles and wind is shown in Figure~\ref{fig:dropgraph}.
\section{Payload Planner}

The payload path planner is calculated using two distinct regions.


\AUVSIFigure
{./figs/dropgraph.png}
{\textwidth}
{Payload release planning with boundaries and obstacles in blue and payload movement in red}
{fig:dropgraph}

\subsection{Commanded Release to Parachute Open}

The first region is between when the command to release is given to the servo to when the parachute is fully open. Assumptions:

    Time delay between the command release to the servo and the bay door opening is constant and known
    Time delay between the bay door opening is constant and known (through experimentation)
    Height difference is calculated between when the bay door opens and when the parachute opens
    Wind is steady state (no gusts)
    The only force acting on the payload is gravity (no aerodynamic drag, etc.)

\subsection{Parachute Open to Target}

The second region is between when the parachute opens to when the payload hits the ground target Assumptions:

    Payload descends down at a constant rate which is known (through experimentation)
    Payload moves in the North and East directions at the speed of the wind
    No aerodynamic drag, acceleration

\subsection{Supporting Waypoints}

The final step of the payload planner is to create supporting waypoints (green triangles) so that the plane is flying in a straight line when it drops the payload.
The planner first tries to fly directly into the wind. If that commanded chi angle would hit an obstacle or go out of bounds, it iterates on the command chi angle by adding 15 degrees until it finds a successful waypoint path.

\end{document}
