%
% API Documentation for AUVSI Imaging Server
% Module src.dao.model.point
%
% Generated by epydoc 3.0.1
% [Mon Dec 10 18:31:41 2018]
%

%%%%%%%%%%%%%%%%%%%%%%%%%%%%%%%%%%%%%%%%%%%%%%%%%%%%%%%%%%%%%%%%%%%%%%%%%%%
%%                          Module Description                           %%
%%%%%%%%%%%%%%%%%%%%%%%%%%%%%%%%%%%%%%%%%%%%%%%%%%%%%%%%%%%%%%%%%%%%%%%%%%%

    \index{src \textit{(package)}!src.dao \textit{(package)}!src.dao.model \textit{(package)}!src.dao.model.point \textit{(module)}|(}
\section{Module src.dao.model.point}

    \label{src:dao:model:point}

%%%%%%%%%%%%%%%%%%%%%%%%%%%%%%%%%%%%%%%%%%%%%%%%%%%%%%%%%%%%%%%%%%%%%%%%%%%
%%                               Variables                               %%
%%%%%%%%%%%%%%%%%%%%%%%%%%%%%%%%%%%%%%%%%%%%%%%%%%%%%%%%%%%%%%%%%%%%%%%%%%%

  \subsection{Variables}

    \vspace{-1cm}
\hspace{\varindent}\begin{longtable}{|p{\varnamewidth}|p{\vardescrwidth}|l}
\cline{1-2}
\cline{1-2} \centering \textbf{Name} & \centering \textbf{Description}& \\
\cline{1-2}
\endhead\cline{1-2}\multicolumn{3}{r}{\small\textit{continued on next page}}\\\endfoot\cline{1-2}
\endlastfoot\raggedright \_\-\_\-p\-a\-c\-k\-a\-g\-e\-\_\-\_\- & \raggedright \textbf{Value:} 
{\tt \texttt{'}\texttt{src.dao.model}\texttt{'}}&\\
\cline{1-2}
\end{longtable}


%%%%%%%%%%%%%%%%%%%%%%%%%%%%%%%%%%%%%%%%%%%%%%%%%%%%%%%%%%%%%%%%%%%%%%%%%%%
%%                           Class Description                           %%
%%%%%%%%%%%%%%%%%%%%%%%%%%%%%%%%%%%%%%%%%%%%%%%%%%%%%%%%%%%%%%%%%%%%%%%%%%%

    \index{src \textit{(package)}!src.dao \textit{(package)}!src.dao.model \textit{(package)}!src.dao.model.point \textit{(module)}!src.dao.model.point.point \textit{(class)}|(}
\subsection{Class point}

    \label{src:dao:model:point:point}
Represents a point datatype from postgres. Used by manual\_cropped model 
for crop\_coordinates


%%%%%%%%%%%%%%%%%%%%%%%%%%%%%%%%%%%%%%%%%%%%%%%%%%%%%%%%%%%%%%%%%%%%%%%%%%%
%%                                Methods                                %%
%%%%%%%%%%%%%%%%%%%%%%%%%%%%%%%%%%%%%%%%%%%%%%%%%%%%%%%%%%%%%%%%%%%%%%%%%%%

  \subsubsection{Methods}

    \label{src:dao:model:point:point:__init__}
    \index{src \textit{(package)}!src.dao \textit{(package)}!src.dao.model \textit{(package)}!src.dao.model.point \textit{(module)}!src.dao.model.point.point \textit{(class)}!src.dao.model.point.point.\_\_init\_\_ \textit{(method)}}

    \vspace{0.5ex}

\hspace{.8\funcindent}\begin{boxedminipage}{\funcwidth}

    \raggedright \textbf{\_\_init\_\_}(\textit{self}, \textit{ptStr}={\tt None}, \textit{x}={\tt None}, \textit{y}={\tt None})

    \vspace{-1.5ex}

    \rule{\textwidth}{0.5\fboxrule}
\setlength{\parskip}{2ex}
    Provides various ways to initialize different point types

\setlength{\parskip}{1ex}
      \textbf{Parameters}
      \vspace{-1ex}

      \begin{quote}
        \begin{Ventry}{xxxxx}

          \item[ptStr]

          String of a integer point, should look something like: "(45,56)"

            {\it (type=string)}

          \item[x]

          Integer for the x component of the point

            {\it (type=int)}

          \item[y]

          Integer for the y component of the point

            {\it (type=int)}

        \end{Ventry}

      \end{quote}

    \end{boxedminipage}

    \label{src:dao:model:point:point:toSql}
    \index{src \textit{(package)}!src.dao \textit{(package)}!src.dao.model \textit{(package)}!src.dao.model.point \textit{(module)}!src.dao.model.point.point \textit{(class)}!src.dao.model.point.point.toSql \textit{(method)}}

    \vspace{0.5ex}

\hspace{.8\funcindent}\begin{boxedminipage}{\funcwidth}

    \raggedright \textbf{toSql}(\textit{self})

    \vspace{-1.5ex}

    \rule{\textwidth}{0.5\fboxrule}
\setlength{\parskip}{2ex}
    Generate a string that can be successfully inserted as a point into 
    postgres. Requires both x and y attributes to be present.

\setlength{\parskip}{1ex}
      \textbf{Return Value}
    \vspace{-1ex}

      \begin{quote}
      String representing the point. Formatted: (x,y). If x or y is not 
      present, None.

      {\it (type=string)}

      \end{quote}

    \end{boxedminipage}

    \label{src:dao:model:point:point:toDict}
    \index{src \textit{(package)}!src.dao \textit{(package)}!src.dao.model \textit{(package)}!src.dao.model.point \textit{(module)}!src.dao.model.point.point \textit{(class)}!src.dao.model.point.point.toDict \textit{(method)}}

    \vspace{0.5ex}

\hspace{.8\funcindent}\begin{boxedminipage}{\funcwidth}

    \raggedright \textbf{toDict}(\textit{self})

    \vspace{-1.5ex}

    \rule{\textwidth}{0.5\fboxrule}
\setlength{\parskip}{2ex}
    Return attributes contained in this model as a dictionary

\setlength{\parskip}{1ex}
      \textbf{Return Value}
    \vspace{-1ex}

      \begin{quote}
      String dictionary of point properties. If x or y is not present, None

      {\it (type=\{int\})}

      \end{quote}

    \end{boxedminipage}

    \label{src:dao:model:point:point:__str__}
    \index{src \textit{(package)}!src.dao \textit{(package)}!src.dao.model \textit{(package)}!src.dao.model.point \textit{(module)}!src.dao.model.point.point \textit{(class)}!src.dao.model.point.point.\_\_str\_\_ \textit{(method)}}

    \vspace{0.5ex}

\hspace{.8\funcindent}\begin{boxedminipage}{\funcwidth}

    \raggedright \textbf{\_\_str\_\_}(\textit{self})

    \vspace{-1.5ex}

    \rule{\textwidth}{0.5\fboxrule}
\setlength{\parskip}{2ex}
    Debug convenience method to get this instance as a string

\setlength{\parskip}{1ex}
    \end{boxedminipage}


%%%%%%%%%%%%%%%%%%%%%%%%%%%%%%%%%%%%%%%%%%%%%%%%%%%%%%%%%%%%%%%%%%%%%%%%%%%
%%                              Properties                               %%
%%%%%%%%%%%%%%%%%%%%%%%%%%%%%%%%%%%%%%%%%%%%%%%%%%%%%%%%%%%%%%%%%%%%%%%%%%%

  \subsubsection{Properties}

    \vspace{-1cm}
\hspace{\varindent}\begin{longtable}{|p{\varnamewidth}|p{\vardescrwidth}|l}
\cline{1-2}
\cline{1-2} \centering \textbf{Name} & \centering \textbf{Description}& \\
\cline{1-2}
\endhead\cline{1-2}\multicolumn{3}{r}{\small\textit{continued on next page}}\\\endfoot\cline{1-2}
\endlastfoot\raggedright x\- & \raggedright X component of the point&\\
\cline{1-2}
\raggedright y\- & \raggedright Y component of the point&\\
\cline{1-2}
\end{longtable}


%%%%%%%%%%%%%%%%%%%%%%%%%%%%%%%%%%%%%%%%%%%%%%%%%%%%%%%%%%%%%%%%%%%%%%%%%%%
%%                            Class Variables                            %%
%%%%%%%%%%%%%%%%%%%%%%%%%%%%%%%%%%%%%%%%%%%%%%%%%%%%%%%%%%%%%%%%%%%%%%%%%%%

  \subsubsection{Class Variables}

    \vspace{-1cm}
\hspace{\varindent}\begin{longtable}{|p{\varnamewidth}|p{\vardescrwidth}|l}
\cline{1-2}
\cline{1-2} \centering \textbf{Name} & \centering \textbf{Description}& \\
\cline{1-2}
\endhead\cline{1-2}\multicolumn{3}{r}{\small\textit{continued on next page}}\\\endfoot\cline{1-2}
\endlastfoot\raggedright I\-N\-T\-\_\-R\-E\-G\-E\-X\- & \raggedright \textbf{Value:} 
{\tt \texttt{'}\texttt{[{\textasciicircum}{\textbackslash}{\textbackslash}d]*({\textbackslash}{\textbackslash}d+)[{\textasciicircum}{\textbackslash}{\textbackslash}d]*}\texttt{'}}&\\
\cline{1-2}
\end{longtable}

    \index{src \textit{(package)}!src.dao \textit{(package)}!src.dao.model \textit{(package)}!src.dao.model.point \textit{(module)}!src.dao.model.point.point \textit{(class)}|)}
    \index{src \textit{(package)}!src.dao \textit{(package)}!src.dao.model \textit{(package)}!src.dao.model.point \textit{(module)}|)}
