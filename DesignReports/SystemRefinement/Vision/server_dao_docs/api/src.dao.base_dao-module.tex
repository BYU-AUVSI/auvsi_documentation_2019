%
% API Documentation for AUVSI Imaging Server
% Module src.dao.base_dao
%
% Generated by epydoc 3.0.1
% [Mon Dec 10 18:31:41 2018]
%
\section{Introduction}
This document provides detailed methods of the Imaging server's database layer. Detailed 
explanations of the input parameters and return types of all methods are given. This document
will be most useful to developers hoping to better understand the Imaging server and possibly
modify it's codebase.

Note that the REST API modules (contained in the src/apis/ directory of the server) are not documented
here. These modules are automatically documented by the Swagger toolchain. When the server is running, you
can navigate to its homepage (localhost:5000 if running on your machine), and use the interactive
documentation there to understand and try the various REST API methods. 

%%%%%%%%%%%%%%%%%%%%%%%%%%%%%%%%%%%%%%%%%%%%%%%%%%%%%%%%%%%%%%%%%%%%%%%%%%%
%%                          Module Description                           %%
%%%%%%%%%%%%%%%%%%%%%%%%%%%%%%%%%%%%%%%%%%%%%%%%%%%%%%%%%%%%%%%%%%%%%%%%%%%

    \index{src \textit{(package)}!src.dao \textit{(package)}!src.dao.base\_dao \textit{(module)}|(}
\section{Module src.dao.base\_dao}

    \label{src:dao:base_dao}

%%%%%%%%%%%%%%%%%%%%%%%%%%%%%%%%%%%%%%%%%%%%%%%%%%%%%%%%%%%%%%%%%%%%%%%%%%%
%%                           Class Description                           %%
%%%%%%%%%%%%%%%%%%%%%%%%%%%%%%%%%%%%%%%%%%%%%%%%%%%%%%%%%%%%%%%%%%%%%%%%%%%

    \index{src \textit{(package)}!src.dao \textit{(package)}!src.dao.base\_dao \textit{(module)}!src.dao.base\_dao.BaseDAO \textit{(class)}|(}
\subsection{Class BaseDAO}

    \label{src:dao:base_dao:BaseDAO}
\begin{tabular}{cccccc}
% Line for object, linespec=[False]
\multicolumn{2}{r}{\settowidth{\BCL}{object}\multirow{2}{\BCL}{object}}
&&
  \\\cline{3-3}
  &&\multicolumn{1}{c|}{}
&&
  \\
&&\multicolumn{2}{l}{\textbf{src.dao.base\_dao.BaseDAO}}
\end{tabular}

DAO with basic methods. All other DAO's are child classes of BaseDAO. 
Initializes and contains a postgres connection object when created.


%%%%%%%%%%%%%%%%%%%%%%%%%%%%%%%%%%%%%%%%%%%%%%%%%%%%%%%%%%%%%%%%%%%%%%%%%%%
%%                                Methods                                %%
%%%%%%%%%%%%%%%%%%%%%%%%%%%%%%%%%%%%%%%%%%%%%%%%%%%%%%%%%%%%%%%%%%%%%%%%%%%

  \subsubsection{Methods}

    \vspace{0.5ex}

\hspace{.8\funcindent}\begin{boxedminipage}{\funcwidth}

    \raggedright \textbf{\_\_init\_\_}(\textit{self}, \textit{configFilePath}={\tt '../conf/config.ini'})

    \vspace{-1.5ex}

    \rule{\textwidth}{0.5\fboxrule}
\setlength{\parskip}{2ex}
    Startup the DAO. Attempts to connect to the postgresql database using 
    the settings specified in the confg.ini file

\setlength{\parskip}{1ex}
      Overrides: object.\_\_init\_\_

    \end{boxedminipage}

    \label{src:dao:base_dao:BaseDAO:close}
    \index{src \textit{(package)}!src.dao \textit{(package)}!src.dao.base\_dao \textit{(module)}!src.dao.base\_dao.BaseDAO \textit{(class)}!src.dao.base\_dao.BaseDAO.close \textit{(method)}}

    \vspace{0.5ex}

\hspace{.8\funcindent}\begin{boxedminipage}{\funcwidth}

    \raggedright \textbf{close}(\textit{self})

    \vspace{-1.5ex}

    \rule{\textwidth}{0.5\fboxrule}
\setlength{\parskip}{2ex}
    Safely close the DAO's connection. It is higly recommended you call 
    this method before finishing with a dao.

\setlength{\parskip}{1ex}
    \end{boxedminipage}

    \label{src:dao:base_dao:BaseDAO:conn}
    \index{src \textit{(package)}!src.dao \textit{(package)}!src.dao.base\_dao \textit{(module)}!src.dao.base\_dao.BaseDAO \textit{(class)}!src.dao.base\_dao.BaseDAO.conn \textit{(method)}}

    \vspace{0.5ex}

\hspace{.8\funcindent}\begin{boxedminipage}{\funcwidth}

    \raggedright \textbf{conn}(\textit{self}, \textit{conn})

\setlength{\parskip}{2ex}
\setlength{\parskip}{1ex}
    \end{boxedminipage}

    \label{src:dao:base_dao:BaseDAO:getResultingId}
    \index{src \textit{(package)}!src.dao \textit{(package)}!src.dao.base\_dao \textit{(module)}!src.dao.base\_dao.BaseDAO \textit{(class)}!src.dao.base\_dao.BaseDAO.getResultingId \textit{(method)}}

    \vspace{0.5ex}

\hspace{.8\funcindent}\begin{boxedminipage}{\funcwidth}

    \raggedright \textbf{getResultingId}(\textit{self}, \textit{stmt}, \textit{values})

    \vspace{-1.5ex}

    \rule{\textwidth}{0.5\fboxrule}
\setlength{\parskip}{2ex}
    Get the first id returned from a statement. Basically this assumes you 
    have a 'RETURNING id;' at the end of the query you are executing 
    (insert or update)

\setlength{\parskip}{1ex}
      \textbf{Parameters}
      \vspace{-1ex}

      \begin{quote}
        \begin{Ventry}{xxxxxx}

          \item[stmt]

          The sql statement string to execute

            {\it (type=string)}

          \item[values]

          Ordered list of values to place in the statement

            {\it (type=list)}

        \end{Ventry}

      \end{quote}

    \end{boxedminipage}

    \label{src:dao:base_dao:BaseDAO:executeStatements}
    \index{src \textit{(package)}!src.dao \textit{(package)}!src.dao.base\_dao \textit{(module)}!src.dao.base\_dao.BaseDAO \textit{(class)}!src.dao.base\_dao.BaseDAO.executeStatements \textit{(method)}}

    \vspace{0.5ex}

\hspace{.8\funcindent}\begin{boxedminipage}{\funcwidth}

    \raggedright \textbf{executeStatements}(\textit{self}, \textit{stmts})

    \vspace{-1.5ex}

    \rule{\textwidth}{0.5\fboxrule}
\setlength{\parskip}{2ex}
    Tries to execute all SQL statements in the stmts list. These will be 
    performed in a single transaction. Returns nothing, so useless if 
    you're trying to execute a series of fetches

\setlength{\parskip}{1ex}
      \textbf{Parameters}
      \vspace{-1ex}

      \begin{quote}
        \begin{Ventry}{xxxxx}

          \item[stmts]

          List of sql statements to execute

            {\it (type=[string])}

        \end{Ventry}

      \end{quote}

    \end{boxedminipage}

    \label{src:dao:base_dao:BaseDAO:basicTopSelect}
    \index{src \textit{(package)}!src.dao \textit{(package)}!src.dao.base\_dao \textit{(module)}!src.dao.base\_dao.BaseDAO \textit{(class)}!src.dao.base\_dao.BaseDAO.basicTopSelect \textit{(method)}}

    \vspace{0.5ex}

\hspace{.8\funcindent}\begin{boxedminipage}{\funcwidth}

    \raggedright \textbf{basicTopSelect}(\textit{self}, \textit{stmt}, \textit{values})

    \vspace{-1.5ex}

    \rule{\textwidth}{0.5\fboxrule}
\setlength{\parskip}{2ex}
    Gets the first (top) row of the given select statement.

\setlength{\parskip}{1ex}
      \textbf{Parameters}
      \vspace{-1ex}

      \begin{quote}
        \begin{Ventry}{xxxxxx}

          \item[stmt]

          Sql statement string to run

            {\it (type=string)}

          \item[values]

          List of objects (generally int, float and string), to safely 
          place in the sql statement.

            {\it (type=[object])}

        \end{Ventry}

      \end{quote}

      \textbf{Return Value}
    \vspace{-1ex}

      \begin{quote}
      The first row of the select stmt's result. If the statement fails or 
      does not retrieve any records, None is returned.

      {\it (type=[string])}

      \end{quote}

    \end{boxedminipage}


\large{\textbf{\textit{Inherited from object}}}

\begin{quote}
\_\_delattr\_\_(), \_\_format\_\_(), \_\_getattribute\_\_(), \_\_hash\_\_(), \_\_new\_\_(), \_\_reduce\_\_(), \_\_reduce\_ex\_\_(), \_\_repr\_\_(), \_\_setattr\_\_(), \_\_sizeof\_\_(), \_\_str\_\_(), \_\_subclasshook\_\_()
\end{quote}

%%%%%%%%%%%%%%%%%%%%%%%%%%%%%%%%%%%%%%%%%%%%%%%%%%%%%%%%%%%%%%%%%%%%%%%%%%%
%%                              Properties                               %%
%%%%%%%%%%%%%%%%%%%%%%%%%%%%%%%%%%%%%%%%%%%%%%%%%%%%%%%%%%%%%%%%%%%%%%%%%%%

  \subsubsection{Properties}

    \vspace{-1cm}
\hspace{\varindent}\begin{longtable}{|p{\varnamewidth}|p{\vardescrwidth}|l}
\cline{1-2}
\cline{1-2} \centering \textbf{Name} & \centering \textbf{Description}& \\
\cline{1-2}
\endhead\cline{1-2}\multicolumn{3}{r}{\small\textit{continued on next page}}\\\endfoot\cline{1-2}
\endlastfoot\multicolumn{2}{|l|}{\textit{Inherited from object}}\\
\multicolumn{2}{|p{\varwidth}|}{\raggedright \_\_class\_\_}\\
\cline{1-2}
\end{longtable}

    \index{src \textit{(package)}!src.dao \textit{(package)}!src.dao.base\_dao \textit{(module)}!src.dao.base\_dao.BaseDAO \textit{(class)}|)}
    \index{src \textit{(package)}!src.dao \textit{(package)}!src.dao.base\_dao \textit{(module)}|)}
