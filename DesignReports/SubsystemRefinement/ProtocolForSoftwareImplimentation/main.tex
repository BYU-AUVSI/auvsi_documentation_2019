\documentclass[]{auvsi_doc}
\setkeys{auvsi_doc.cls}{
	AUVSITitle={Protocol for Software Implementation},
	AUVSIRevision=0.0,
	AUVSIDescription={Created},
	AUVSIAuthor={Kameron Eves},
	AUVSIChecker={[Checker]},
	AUVSILogoPath={./figs/logo.pdf},
	AUVSIDocID={AF-012}
}

% include extra packages, if needed
\usepackage{makecell}
\usepackage{longtable}
\usepackage{hyperref}

% Remove Heading Numbers
\setcounter{secnumdepth}{0}

\begin{document}
\begin{AUVSITitlePage}
\begin{artifacttable}
	\entry{AF-012, 0.1, 3-19-2018, Created, Kameron Eves, [CHECKER]}
\end{artifacttable}
\end{AUVSITitlePage}
% document contents
\section{Introduction}

The purpose of this document is to detail the protocol for implementing new software on the aircraft. A new branch on git has been created which can only be changed via pull request. Only team leads can merge pull requests. Below are several different ways to make changes on the plane.

Some of the methods described below require you to install the hub software. This software allows command line shortcuts for communicating with Git Hub. It is not required. There is always another way to do each action, but it may require more work. Instructions to install the hub software along with the source code can be found at: \url{https://github.com/github/hub}  

All command line instructions assume that the current working directory is the repository you are working with. All instructions also assume that the your changes are committed, pushed, and you've pulled the other most recent changes.

\section{To Update Software on Aircraft}
This is to make permanent changes to the aircraft's code.
\begin{enumerate}
	\item \textbf{Write and test your changes on a branch other then the plane branch.} When you begin making your changes, ensure that your branch is based off of the most recent commit of the plane branch. You can do this by making a new brach for your changes, or merging the plane branch into the branch you want to work on. On your computer, either:
	\begin{enumerate}
		\item git checkout plane
		\item git branch \textless branch name\textgreater
		\item git checkout \textless branch name\textgreater
	\end{enumerate}
	or 
	\begin{enumerate}
		\item git checkout \textless branch name\textgreater
		\item git merge plane
	\end{enumerate}
	\item \textbf{Create a pull request.} Either:
	\begin{enumerate}
		\item open www.github.com
		\item navigate to the repository you want to change.
		\item select "New Pull Request"
		\item select your branch on the right
		\item select the plane branch on the left (the arrow should be pointing at the plane branch)
		\begin{itemize}
			\item NOTE: Ensure that these are from the BYU-AUVSI copies of these repositories. Do not merged into the repositories our work is forked from.
		\end{itemize}
		\item select "Create Pull Request"
	\end{enumerate}
	or (this requires hub)
	\begin{enumerate}
		\item git checkout \textless branch name\textgreater
		\item send\_to\_plane "\textless Description of changes\textgreater"
	\end{enumerate}
	\item \textbf{Inform your team lead you have created a pull request they need to review.}
	\item \textbf{After pull request is approved on git hub}
	\begin{enumerate}
		\item ssh into plane
		\item cd to the relevant directory
		\item git pull
		\item cd to top of workspace
		\item catkin\_make
	\end{enumerate}
\end{enumerate}

\section{Test Your Code on Aircraft}
This is for temporarily testing your code on the aircraft
\begin{enumerate}
	\item ssh into plane
	\item cd to the relevant directory
	\item git pull
	\item git checkout \textless branch name\textgreater
	\item cd to top of workspace
	\item catkin\_make
	\item run tests
	\item commit and push any changes you make during test
	\item \textbf{git checkout plane}
	\item cd to top of workspace
	\item catkin\_make
\end{enumerate}

\section{Emergency Changes}
This is for changes that must happen right now (i.e. emergencies during flight test) and should be very small changes.
\begin{enumerate}
	\item ssh into plane
	\item cd to the relevant directory
	\item open files in command line editor
	\item make changes
	\item cd to top of workspace
	\item catkin\_make
	\item test
	\item \textbf{commit and push any changes you make}
\end{enumerate}

\section{Add a New Repository}
To add a new package to the aircraft speak with Kameron. Also note that this protocol will not protect individual sub team's code (i.e. code on UGV, ground stations, etc...) if you speak with Kameron he can set this up for your code as well if you'd like.

\end{document}
