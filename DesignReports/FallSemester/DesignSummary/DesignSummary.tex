\documentclass[]{auvsi_doc}
\setkeys{auvsi_doc.cls}{
	AUVSITitle={Design Summary},
%	AUVSIRevision=0.0,
%	AUVSIDescription={Created},
%	AUVSIAuthor={Kameron Eves},
%	AUVSIChecker={[Checker]},
	AUVSILogoPath={./figs/logo.pdf},
%	AUVSIDocID={AF-004}
}

% include extra packages, if needed
\usepackage{makecell}

% Remove Heading Numbers
\setcounter{secnumdepth}{0}

\begin{document}
\begin{AUVSITitlePage}
\begin{artifacttable}
	\entry{TW-002, 0.1, 12-03-2018, Created, Kameron Eves, [Checker]}
\end{artifacttable}
\end{AUVSITitlePage}

\section{Introduction}
\subsection{Project Description}

Each year, the Association for Unmanned Vehicle Systems International (AUVSI) hosts a Student Unmanned Aerial Systems (SUAS) competition. While each year’s competition has unique challenges, the general challenge is to build an Unmanned Aerial System (UAS) capable of autonomous flight, object detection, and payload delivery. This year’s competition will be held June 12th to 15th, 2019 at the Naval Air Station in Patuxent River, Maryland.

The UAS's entered into the competition are judged primarily on their mission success during the competition. This year's mission begins when the team hands control of the aircraft to the autopilot. The autopilot will then initiate the takeoff and attempt to execute the following tasks.

\begin{itemize}
	\item\textbf{Fly Waypoint Path} - Fly waypoints given to the the team just prior to the competition. In this process, and throughout the entire mission, the aircraft must avoid virtual obstacles and stay within boundaries (both horizontal and vertical).
	\item\textbf{Visual Target Classification} - Within a prescribed search zone there will be a number of large cardboard shapes with an alpha-numeric character on them. The aircraft must capture an image of these shaps
	\item\textbf{Payload Delivery} -
\end{itemize}

In general, every aspect of the mission should be completed autonomously.

For the last two years BYU has sponsored an AUVSI team to compete in the competition. The 2017 team was primarily volunteer based and placed 10th overall while the 2018 team was a Capstone team and placed 9th overall. This year’s team is also a Capstone team consisting of BYU Mechanical, Electrical, and Computer Engineering students and looks to place as one of the top five teams.
\section{Description of Design}
\subsection{Airframe}
\subsection{Visual Target Classification}
\subsection{Payload Delivery}
\section{Summary of Expected Performance}
\subsection{Airframe}
\subsection{Visual Target Classification}
\subsection{Payload Delivery}
\section{Status and Future Plans}
\subsection{Airframe}
\subsection{Visual Target Classification}
\subsection{Payload Delivery}
\section{Conclusion}

% document contents
\end{document}
