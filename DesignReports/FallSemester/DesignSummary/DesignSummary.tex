\documentclass[]{auvsi_doc}
\setkeys{auvsi_doc.cls}{
	AUVSITitle={Design Summary},
%	AUVSIRevision=0.0,
%	AUVSIDescription={Created},
%	AUVSIAuthor={Kameron Eves},
%	AUVSIChecker={[Checker]},
	AUVSILogoPath={./figs/logo.pdf},
%	AUVSIDocID={AF-004}
}

% include extra packages, if needed
\usepackage{makecell}
\usepackage{multirow}

% Remove Heading Numbers
\setcounter{secnumdepth}{0}

\begin{document}
\begin{AUVSITitlePage}
\begin{artifacttable}
	\entry{TW-002, 0.1, 12-03-2018, Created, Kameron Eves, [Checker]}
\end{artifacttable}
\end{AUVSITitlePage}


\section{ Introduction}

Each year, the Association for Unmanned Vehicle Systems International (AUVSI) hosts a Student Unmanned Aerial Systems (SUAS) competition. This year’s competition will be held June 12th to 15th, 2019 and BYU will be sending a team to compete.

The aircraft entered into the competition are judged primarily on a demonstration of ability to autonomously complete a mission which includes the following tasks:

\begin{itemize}
	\item\textbf{Fly Waypoint Path} - Fly waypoints given to the the team just prior to the competition. In this process, and throughout the entire mission, the aircraft must avoid virtual obstacles and stay within boundaries (both horizontal and vertical).
	\item\textbf{Visual Target Classification} - Capture an image of several targets within a search area and report to the judges the shape, color, alpha-numeric character, alpha-numeric character color, and geolocation of each target.
	\item\textbf{Payload Delivery} - Drop on a specified location, an unmanned ground vehicle (UGV) that itself carries a small water bottle. Then, carrying its the water bottle, the UGV must drive to a second specified location.
\end{itemize}

In order, to accomplish these tasks our team has decided upon the following objective statement: 
 
\begin{quote}
Improve upon last year’s BYU AUVSI unmanned aerial system (UAS) by improving path planning, obstacle avoidance, visual object detection, and payload delivery by April 1, 2019 with a budget of \$3,500 and 2,500 man hours.
\end{quote}

We have stipulated several key success measures which are enumerated in Table~\ref{tab:key_measures}. Additional market requirements, performance measures, ideal values, and target values are included in RM-001.

\begin{table}[H]
	\centering
	\caption{Key success measures for the UAS}\label{tab:key_measures}
\begin{tabular}{|P{7.1cm}|>{\centering\arraybackslash}P{0.75cm}|>{\centering\arraybackslash}P{0.75cm}>{\centering\arraybackslash}P{0.75cm}>{\centering\arraybackslash}P{0.75cm}|>{\centering\arraybackslash}P{0.75cm}>{\centering\arraybackslash}P{0.75cm}>{\centering\arraybackslash}P{0.75cm}|}
	\multicolumn{8}{c}{\textbf{\underline{Key}}} \\
		\multicolumn{8}{l}{\textbf{Performance:} \textbf{SG} = Stretch Goal, \textbf{A} = Excellent,	\textbf{B} = Good,	\textbf{C} = Fair}  \\
		\multicolumn{8}{l}{\textbf{Acceptable:} \textbf{L} = Lower, \textbf{I} = Ideal,	\textbf{U} = Upper}\\
	\hline
	\rowcolor[HTML]{C0C0C0}
	 & \multicolumn{4}{c|}{\textbf{Performance}} &\multicolumn{3}{c|}{\textbf{Acceptable}}  \\\rowcolor[HTML]{C0C0C0} 
	\multirow{-2}{*}{ \textbf{Measures (units)}} &{\textbf{SG}} & {\textbf{A}} & {\textbf{B}} & {\textbf{C}} & {\textbf{L}} & {\textbf{I}} & {\textbf{U}} \\
	
	\hline
	\textbf{Obstacles Hit (\#)} & 0 & 1 & 3 & 5 & 0 & 0 & 5 \\
	\hline
	\textbf{Average Waypoint Proximity (ft)} & 5 & 20 & 25 & 30 & 0 & 0 & 100 \\
	\hline
	\textbf{Characteristics Identified (\%)} & 80 & 40 & 30 & 20 & 20 & 100 & 100 \\
	\hline
	\textbf{Airdrop Accuracy (ft)} & 5 & 25 & 50 & 75 & 0 & 0 & 75 \\
	\hline
	\textbf{Number of Manual Takeovers (\#)} & 0 & 1 & 2 & 3 & 0 & 0 & 3 \\
	\hline
\end{tabular}
\end{table}


\section{Description of Design}
\subsection{Airframe}
\subsection{Visual Target Classification}
\subsection{Payload Delivery}
\section{Summary of Expected Performance}
\subsection{Airframe}
\subsection{Visual Target Classification}
\subsection{Payload Delivery}
\section{Status and Future Plans}
\subsection{Airframe}
\subsection{Visual Target Classification}
\subsection{Payload Delivery}
\section{Conclusion}

% document contents
\end{document}
