\documentclass[]{auvsi_doc}
\setkeys{auvsi_doc.cls}{
	AUVSITitle={UAS Subsystem Testing},
	AUVSILogoPath={./figs/logo.pdf}
}

\usepackage{longtable}

\begin{document}

\begin{AUVSITitlePage}
\begin{artifacttable}
\entry{SS-002, 0.1, 10-29-2018, initial draft, Andrew Torgesen, Derek Knowles}
\entry{SS-002, 1.0, 10-31-2018, pre-design review revisions, Andrew Torgesen, Tyler Miller}
\entry{SS-002, 1.1, 11-08-2018, after design review revisions, Kameron Eves, [checker]}
\end{artifacttable}
\end{AUVSITitlePage}

\section{Motivation}

As described in the UAS Subsystem Interface Definition document (SS-001), there are two main data links between the aircraft and the subsystems on the ground during a competition flight:

\begin{itemize}
	\item The \textbf{900 MHz Radio Link} between the RC transmitter and receiver constitutes the minimal level of communication necessary for flight. The RC link allows a safety pilot to arm/disarm the aircraft's throttle and toggle the autopilot. As stipulated in our key success measures, we are trying to minimize instances when the safety pilot must manually take over the aircraft via the RC link. The ideal flight would not utilize this communication method. However, the RC link is necessary for safe operation and as such is essential to our product. If RC is lost, then the autopilot should immediately activate a \textit{failsafe} mode.
	\item The \textbf{Ubiquiti WiFi Link} between the Ubiquiti Rocket (on the ground) and Bullet (on the aircraft) allows for the exchanging of data over a ROS network. Effectively, the Rocket and the Bullet allow for network connectivity between all subsystems on the ground and in the air. For example, within our key success measures, target characteristics identified pre-requires the ability to communicate images with the ground station. The aircraft's proximity to waypoints and obstacles are also reported through this data link. Thus the Ubiquiti WiFi link will be essential to a successful performance in our key success measures.
\end{itemize}

Almost all subsystem interfaces and performance measures depend on these two data links. Outlined in this document are testing procedures and results to evaluate the quality and reliability of each of these vital data links for the UAS system as a whole.

\section{Testing Descriptions and Procedures}

Table \ref{tab:sstests} outlines key characteristics of the WiFi and RC data links that should be tested, as well as how they should be tested.

\begin{center}
	%\captionof{table}{Description of testing procedures for UAS WiFi and RC data links.}
	%\label{tab:sstests}
	\begin{longtable}[H]{|P{3cm}|P{5.75cm}|P{5.75cm}|}
		\caption{Description of testing procedures for UAS WiFi and RC data links.}
		\label{tab:sstests}\\
		\hline
		\rowcolor[HTML]{C0C0C0}
		{\color[HTML]{000000} \textbf{Test name}} & {\color[HTML]{000000}\textbf{Characteristic being tested}}	& {\color[HTML]{000000}\textbf{Procedure}} \\
		\hline
		\textbf{RC failsafe}	& If RC connection is lost, then the flight control software should execute a failsafe mode to avoid an uncontrolled crash. &	While the aircraft's autopilot is active, kill the RC transmitter. Observe what the autopilot does. It should guide the aircraft into a loiter flight. \\
		\hline
		\textbf{Network loss}	& If the network connection between the aircraft and the ground is lost, then the aircraft should still be able to complete the tasks allocated to it until connectivity is regained. &	While the aircraft is flying a mission, point the Ubiquity Rocket away from the aircraft, killing the ground-to-air WiFi connection. There should be no visible deviation of the aircraft from its current mission, and RC the connection should still be active. \\
		\hline
		\textbf{Network reliability}	& The network should be able to connect upon boot-up of all subsystem components. Connection should be robust to external conditions and allow for a satisfactory data transfer rate. &	In an outdoor environment, turn on all subsystem components and ensure that they all connect to the network automatically. Max out the stream rate of the camera to the on-board computer. Activate all subsystems that communicate over the network, and measure data transfer rates--particularly the following: \begin{itemize}
			\item Images should be able to stream over the network at a rate of $\geq 1$ Hz.
			\item UAS state data should be viewable on the ground station machines at a rate of  $\geq 4$ Hz.
			\item JSON data packets should be able to be sent to the interop server at a rate of $\geq 4$ Hz.
		\end{itemize} \\
		\hline
		\textbf{ROS failure}	& If the ROS network fails, then the autopilot can no longer fly the aircraft. The safety pilot should be able to take back control of the aircraft over RC to guide it to safety. &	While the autopilot is running, kill the ROS network on the aircraft's on-board computer with ssh. RC connectivity should still be active, and the safety pilot should theoretically be able to control the aircraft well enough to either recover the vehicle or prevent causing harm to surroundings as it crashes. \\
		\hline
	\end{longtable}
\end{center}

\section{Testing Results}

Table \ref{tab:ssresults} gives the results of testing according to the procedures outlined in Table \ref{tab:sstests}, as well as conclusions drawn from those results.

\begin{center}
	% \captionof{table}{Test results for the evaluation of the UAS WiFi and RC data links.}
	% \label{tab:ssresults}
	\begin{longtable}[H]{|P{3cm}|P{5.75cm}|P{5.75cm}|}
		\caption{Test results for the evaluation of the UAS WiFi and RC data links.}
		\label{tab:ssresults}\\
		\hline
		\rowcolor[HTML]{C0C0C0}
		{\color[HTML]{000000} \textbf{Test name}} & {\color[HTML]{000000}\textbf{Test results}}	& {\color[HTML]{000000}\textbf{Conclusions}} \\
		\hline
		\textbf{RC failsafe}	& After RC is lost for $\approx 30s$, the autopilot triggers a ``return to land" protocol, landing near where it took off from. &	The RC failsafe mechanism built into the autopilot has been found to be in line with the AUVSI competition rules. \\
		\hline
		\textbf{Network loss}	& Loss of connection between the Ubiquiti Rocket and Bullet has no discernible impact on the autopilot--the only consequence is that the groundstation computers are unable to view the states of the aircraft over the ROS network. Communication resumes once the aircraft is back in range of the Rocket. &	\begin{itemize}
			\item It will be beneficial to have an on-board state recorder to record all ROS messages for later viewing, even if connection to the aircraft is lost temporarily.
			\item We need to run tests to measure the range of the Rocket/Bullet connection when the Rocket is pointed directly toward the aircraft during flight.
		\end{itemize} \\
		\hline
		\textbf{Network reliability}	& Over the course of numerous flight tests, the network connection starts up reliably in all cases but one. There is a particular spot in a field in Springville where the network will never connect. Moving one block over, the network always connects. \begin{itemize}
			\item \textit{Image stream rate:} 3-4 Hz
			\item \textit{State stream rate:} 40-45 Hz
			\item \textit{JSON stream rate:} 3-4 Hz
		\end{itemize} &	\begin{itemize}
		\item The network streaming rate has been found to be adequate. It is possible that we will want to purchase a more powerful router to allow for faster streaming rates at longer distances.
		\item We have only run the network speed test with the aircraft on the ground; it would be nice to run another speed test in conjunction with a test of the maximum range of the Ubiquiti network connection.
		\item The instance of never being able to connect in a particular geographical location is troubling. This quirk merits further investigation.
	\end{itemize} \\
		\hline
		\textbf{ROS failure}	& The RC connection to the aircraft has been found to be reliable and capable of manual takeover in any situation, as long as the batteries of the transmitter are not depleted. It has been found that certain settings should be toggled on the transmitter to conserve power, otherwise it experiences a battery life of about half an hour, which is inadequate. & \begin{itemize}
			\item The range of the RC connection has been found to be adequate within a radius of $\approx 300ft$.
			\item We should run an additional test to determine the approximate maximum range of the RC connection.
		\end{itemize} \\
		\hline
	\end{longtable}
\end{center}

\section{Conclusion}

Based on the results documented in Table \ref{tab:ssresults}, we have determined that \textbf{our chosen principal inter-component data links are adequate for the competition environment and will not inhibit excellent performance in our key success measures}. Further tests are required to determine the boundary conditions (such as maximum possible distance) of their functional use.

\end{document}
