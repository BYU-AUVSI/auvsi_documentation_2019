\documentclass[]{auvsi_doc}
\setkeys{auvsi_doc.cls}{
	AUVSITitle={Unmanned Ground Vehicle Drop Mechanism Concept Test Procedures and Results},
	AUVSILogoPath={./figs/logo.pdf}
}

% include extra packages, if needed

\begin{document}

\begin{AUVSITitlePage}
\begin{artifacttable}
\entry{GV-004, 0.1, 10-26-2018, Initial creation\, procedures listed, Jacob Willis, Andrew Torgesen}
\entry{GV-004, 1.0, 11-6-2018, Additional detail added based on design review, John Akagi, Andrew Torgesen}

% additional \entry{} commands for extra rows in the revision table, if needed
\end{artifacttable}
\end{AUVSITitlePage}

% document contents (see below for LaTex commands that make your life easier)
\section{Introduction}
This document describes the procedures used to test each of the Unmanned Ground Vehicle (UGV) payload delivery concepts.
Some of the tests were unnecessary for selecting between concepts, so they will not be performed unless required by subsystem engineering.

\section {Test Procedures and Results}
	\subsection{Drop Mechanism Mass}

	The mass of all components related to landing the UGV safely were determined and summed for each concept. Results are found in Table \ref{mass}.


	\begin{table}[!h]
	\centering
	
	\caption{Estimated total mass for the delivery system for the UGV.}
\label{mass}
	\begin{tabular}{|l|l|}
		\hline
		\textbf{Concept}       & \textbf{Result} \\
		\hline
		Parachute              & .026 kg                \\
		Parachute w/ control   & .124 kg                \\
		Skycrane               & .160 kg                \\
		Glider                 & .08 kg				 \\
		\hline
	\end{tabular}
	\end{table}

	\subsection{Maximum Deliverable Weight}

	
	In order to determine the maximum weight the concepts could deliver, the weight constraints of the individual components were determined. The maximum weight is the minimum load ratings. Results are found in Table \ref{weight}.



	\begin{table}[!h]
\centering

\caption{Maximum weight the concept can safely deliver. Weight determined by load ratings of components.}
\label{weight}
	\begin{tabular}{|l|l|}
		\hline
		\textbf{Concept}       & \textbf{Result} \\
		\hline
		Parachute              &         4 kg        \\
		Parachute w/ control   &          4 kg       \\
		Skycrane               &        3 kg         \\
		Glider                 & 		1 kg	 \\
		\hline
	\end{tabular}
	\end{table}

	\subsection{Drop Mechanism Volume}

	The volume of all of the UGV drop mechanisms, and the volume needed for the UGV if the mechanism requires it be inside the aircraft is measured. Results are found in Table \ref{volume}.


	\begin{table}[!h]
\centering

\caption{Volume required for each drop mechanism.}
\label{volume}
	\begin{tabular}{|l|l|}
		\hline
		\textbf{Concept}       & \textbf{Result} \\
		\hline
		Parachute              &        462 cm\textsuperscript{3}         \\
		Parachute w/ control   &    462 cm\textsuperscript{3}             \\
		Skycrane               &          92 cm\textsuperscript{3}       \\
		Glider                 & 		864 cm\textsuperscript{3}	 \\
		\hline
	\end{tabular}
	\end{table}

	\subsection{Stowed Drop Mechanism Drag}                                   


	A preliminary estimate of this is made using the area of the mechanism that is exposed outside of the airframe and computing drag with $D = \frac{1}{2}\rho v^2 C_d A$ where air density $\rho = 1.225 kg/m^3$, velocity $v = 15 m/s$ is the estimated aircraft flight speed, area $A$ is the cross sectional area of the drop mechanism, and $C_d$ is the estimated coefficient of drag based on cross sectional area and standard drag coefficient tables. Results are found in Table \ref{drag}.
	

	\begin{table}[!h]
	\centering

	\caption{Estimated drag of the drop mechanism.}
\label{drag}
	\begin{tabular}{|l|l|}
		\hline
		\textbf{Concept}       & \textbf{Result} \\
		\hline
		Parachute              &        .278 N        \\
		Parachute w/ control   &    	.278 N            \\
		Skycrane               &        .315 N         \\
		Glider                 &		.245 N		 \\
		\hline
	\end{tabular}
	\end{table}
	\subsection{Maximum Landing Velocity}

	
	A preliminary estimate of this is made by calculating the landing velocity based on video data taken during the drop testing. The the payload was compared to a known measure placed behind the payload and the change in position over time was used to calculate the impact velocity. Results are found in Table \ref{vel}.


	\begin{table}[!h]
	\centering

	\caption{Estimated landing velocity of delivery system.}
	\label{vel}
	\begin{tabular}{|l|l|}
		\hline
		\textbf{Concept}       & \textbf{Result} \\
		\hline
		Parachute (48 in)              &        2.7 m/s         \\
		Parachute (30 in)		&	4.8 m/s\\
		Parachute w/ control   &                4.8 m/s \\
		Skycrane               &                 Not Tested\\
		Glider                 &			1.9 m/s	 \\
		\hline
	\end{tabular}
	\end{table}

	\subsection{Delivery Precision}

	A preliminary estimate of this is made by dropping a representative load with the mechanism from a height of 35 feet. The distance between where the load lands and the target is scaled to a 100 foot drop height and the standard deviation of the spread is reported. The precision of the glider was tested by dropping it from heights of 5, 6, and 7 ft and the precision was scaled to 100 ft. For more detailed explanation of the test procedure, see GV-003 UGV Parachute Testing Description. Results are found in Table \ref{precision}.

	\begin{table}[!h]
	\centering

	\caption{Standard deviation of initial impact, scaled to a 100 ft drop.}
	\label{precision}
	\begin{tabular}{|l|l|}
		\hline
		\textbf{Concept}       & \textbf{Result} \\
		\hline
		Parachute (48 in)              &        2.85 ft        \\
		Parachute (30 in)		& 4.14 ft	\\
		Parachute w/ control   &      3.23 ft           \\
		Skycrane               &            Not Tested    \\
		Glider                 &		28 ft		 \\
		\hline
	\end{tabular}
	\end{table}

	\subsection{Rule Violations}


	A checklist of the relevant rules is checked for the concept. The number of violations for the concept is summed. Results are found in Table \ref{rules}.

\subsubsection{UGV Rules Requirements}
The following outline the rules which must be followed in order to achieve any points. 
\begin{itemize}
\item Must carry 8 oz water bottle
\item Must not fly below minimum altitude
\item Must land gently and without damage (subjective measure)
\item Max weight of 48 oz
\end{itemize}


	\begin{table}[!h]
	\centering

	\caption{Number of rules violated by delivery system.}
	\label{rules}
	\begin{tabular}{|l|l|}
		\hline
		\textbf{Concept}       & \textbf{Result} \\
		\hline
		Parachute              &       0          \\
		Parachute w/ control   &       0         \\
		Skycrane               &            0     \\
		Glider                 &		1		 \\
		\hline
	\end{tabular}
	\end{table}

\subsection{Conclusion}
The preceding test results are used to select the optimal concept in GV-002.

\end{document}
