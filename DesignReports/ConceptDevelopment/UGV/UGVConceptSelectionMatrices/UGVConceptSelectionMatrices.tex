\documentclass[]{auvsi_doc}
\setkeys{auvsi_doc.cls}{
	AUVSITitle={[TITLE OF YOUR DOCUMENT]},
	AUVSILogoPath={./figs/logo.pdf}
}

% include extra packages, if needed

\begin{document}

\begin{AUVSITitlePage}
\begin{artifacttable}
\entry{[ARTIFACT ID], [REVISION NUMBER], [DATE], [DESCRIPTION], [AUTHOR], [CHECKED BY]}
% additional \entry{} commands for extra rows in the revision table, if needed
\end{artifacttable}
\end{AUVSITitlePage}

\section{UGV Rules Requirements}
The following outline the rules which must be followed in order to achieve any points. 
\begin{itemize}
	\item Must carry 8 oz water bottle
	\item Must not fly below minimum altitude
	\item Must land gently and without damage (subjective measure)
	\item Max weight of 48 oz
	\item Max speed of 10 mph
	\item UGV must terminate driving after 30 seconds of communication loss or after driving out of the boundary specified
	\item Drive termination must be activated by member of team
	\item No exotic fuels or batteries
	\item Batteries must be brightly colored (bright tape)
	\item The UGV may only drive autonomously
\end{itemize}

\section{Evaluation Methods and Results}

As can be seen from the decision matrix in Table~\ref{cont_cs_tab}, 

\begin{AUVSITable}
{9}
{1.35cm}
{A decision matrix the UGV Drop Method. A scale of 1-5 was used for weights with 5 having high importance and 1 having low importance. A 1-5 scale was also used to rate each option’s performance under each requirement. In this case, a 1 was used to indicate poor performance while a 5 indicates favorable performance.}
{cont_cs_tab}

\entry{UGV Drop Method,Weight,Glider,Sky Crane, Parachute, Un-aided Drop (Reference)}
\entry{UGV Weight,1,0,0,0,0}
\entry{Stowed Drag,1,0,0,0,0}
\entry{Max Drop Height,1,0,0,0,0}
\entry{Max Landing Velocity,1,0,0,0,0}
\entry{Accuracy in Hitting Target,1,0,0,0,0}
\entry{Totals,.,0,0,0,0}

\end{AUVSITable}


% document contents (see below for LaTex commands that make your life easier)

\end{document}
