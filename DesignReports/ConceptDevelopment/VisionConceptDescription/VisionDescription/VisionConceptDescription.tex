\documentclass[]{auvsi_doc}
\setkeys{auvsi_doc.cls}{
	AUVSITitle={Vision Subsystem Concept Definition},
	AUVSILogoPath={./figs/logo.pdf}
}

% include extra packages, if needed

\begin{document}

\begin{AUVSITitlePage}
\begin{artifacttable}
\entry{CD-002, 0.1, 10-25-2018, Initial release, Tyler Miller, Derek Knowles}
% additional \entry{} commands for extra rows in the revision table, if needed
\end{artifacttable}
\end{AUVSITitlePage}

\section{Purpose}

Last year's vision subsystem achieved less than 25\% of possible points related to the subsystem. As such, 
it was determined that major improvements will be made at both the manual and autonomous 
levels.

The competition gives points for correct classification of ground targets' shape, shape color,
alphanumeric, alphanumeric color, alphanumeric orientation, and geolocation. Additional points
are given if the process between taking the image and submitting the classified image to the
judges' server is fully autonomous without the intervention of a human. There is a penalty, 
however, if false positive targets are submitted to the judges' server.

\section{Concept Selected}

Vision's competition requirements are complex and as such required multiple concepts to 
fit into a larger system. After internal discussion, we decided to pursue a base concept 
of manual and autonomous classification systems running in parallel.

\section{Definition}

This year's vision team is changing our system architecture for classifying targets which will
 allow for better communication and organization. Instead of downloading each image and image state
onto someone's personal computer, the computer oboard the plane will send image and vehicle state
data to a server on the ground. This server will have a compiled database of all images captured
and will attach classification data onto each image as it is manually processed. Our 
autonomous detection script will also be querying the server image database and classifying
images. One team member will be monitoring the autonomous output ready to kill the 
program if it is sending too many false positives (which cause the team to incur a 
penalty). Our system architecture is outlined in Figure~\ref{fig:blockfig}.

\AUVSIFigure
{./figs/block.pdf}
{\textwidth}
{Target classification system architecture}
{fig:blockfig}

Our autonomous classification system design is outlined in Figure~\ref{fig:autofig}.
These concepts for autonomous target recognition are based on methods that
other competition teams were able to successfully use at the comptition to
identify targets. We will continue to iterate on the autonomous process, but
we are confident that we can create a reliable and robust system for autonomous
target classification.

\AUVSIFigure
{./figs/auto.pdf}
{\textwidth}
{Autonomous classification system design}
{fig:autofig}

\section{Justification}

Since all of our high-level concepts depend on our imaging hardware, we decided it would be beneficial for us to choose a camera as 
soon as possible. Our list of potential cameras came from previous years systems as well as cameras used by last years top-placing 
teams. Critical performance measures are shown in our measured camera values table (CS-002). This table was directly translated 
into a selection matrix(CS-002). Based off the camera concept selection matrix, it was decided that the Sony a6000 would give us 
the greatest cost to performance. Its large 24MP sensor will improve image quality when flying at higher altitudes and make 
autonomous classification easier. Its auto-stabilization and fast exposure time also remove a lot of burden from the user to adjust 
settings mid-flight. Additionally 7 of the top 15 teams used the a6000 or the earlier generation (but basically equivalent) a5100.

The autonomous classification system is the largest undertaking of this year's vision subteam. Each of the 6 characteristics we are 
required to identify could potentially be done using a different method. Given the high-enumeration of concepts this generates, we 
determined it would be most beneficial for us to select one high level concept which would help define the rest of the system.

Concepts for autonomous classification were formed in three ways. The first was discussing our system requirements with market experts. 
They offered excellent advice on how to best go about the classification problem. The second was researching how top-placing teams from 
previous years tackled the problem. Teams are required to submit a design report which is made publicly available, allowing us understand 
from a high level how their image classification systems worked. Third, we did extensive online research on available software libraries 
and tools that could be used. As we pursued these three methods, our best concept for autonomous classification evolved into its current 
form. We feel that this final concept is the best combination of these three sources.

\end{document}