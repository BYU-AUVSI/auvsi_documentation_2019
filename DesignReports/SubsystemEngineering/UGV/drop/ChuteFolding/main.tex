\documentclass[]{auvsi_doc}
\setkeys{auvsi_doc.cls}{
	AUVSITitle={Parachute Folding},
	AUVSILogoPath={./figs/logo.pdf},
	AUVSIDocID={GV-007}
}

\usepackage{graphicx}
\usepackage{hyperref}
\hypersetup{
    colorlinks=true,
    linkcolor=blue,
    filecolor=blue,      
    urlcolor=blue,
}
 

\begin{document}
\begin{AUVSITitlePage}
\begin{artifacttable} 
	\entry{GV-007, 1.0, 2-1-2019, Created, Jacob Willis, Needs Check}
\end{artifacttable}
\end{AUVSITitlePage}
% document contents


\section*{Introduction}
The parachute is a critical part of the UGV drop system. Ensuring that it unfolds in a repeatable, consistent manner will allow better prediction
of where the UGV will land after deployment. This will allow us to meet our airdrop accuracy key success measure.

\section*{Definitions}
\begin{description}
	\item[Gore] A single fabric piece of the parachute. The alternating red and white pieces.
	\item[Shroud] The strings attached to the bottom of the gores.
	\item[Spill Hole] The hole in the top of the parachute.
	\item[Swivel] The metal coupling at the end of the shrouds.
\end{description}

\section*{Parachute Folding Steps}
A video of the parachute folding process is found in the capstone box folder at: \url{https://byu.box.com/s/l2p32ai9ylidmniiqcs3lx7ycjyawz1b}


\begin{enumerate}
	\item Shake out the parachute and untangle the shrouds until you can separate the two bundles of shrouds. Make sure the shrouds are as untangled as possible.
	\item Pull the shrouds together by starting at the swivel and sliding your hand along the shrouds until you hit the gores. This will pull the gores together so all the shrouds are at the same tension.
	\item Organize the gores to have two even sets of six, one on the right and one on the left. Make sure to pull all of the gores that are folded into the middle out so they lay flat. Also make sure to keep one hand, or a twist tie, holding the shrouds together as you do this.
	\item Lay the parachute on the table, shrouds in the middle and six gores on each side. 
	\item Flatten out the parachute and make everything as flat and even as possible. This is subjective, but make sure there aren't any major wrinkles in the chute.
	\item Using a 5 inch cardboard rectangle, fold the parachute starting from the bottom. There should be four folds, and the folds should end up in a "Z" shape.
	\item Straighten out the shrouds again, making sure they are in a tight, but not tangled bundle at the bottom of the parachute.
	\item Fold the shrouds 3 inches up the folded parachute, then bend them to the side and pull them out straight. It is important to make sure the shrouds don't go over the top of the chute, as this will result in a failure to open.
	\item Roll the chute around the shrouds, keeping the fabric of the chute as uniform as possible.
	\item Continue rolling the shrouds around the chute, trying to keep each line as flat as possible (don't bunch them up) and spiraling down towards the bottom of the chute.
	\item When 6 inches of shroud are left, stop rolling the shrouds.
	\item Keeping the chute tight, slide it into the parachute deployment sleeve, keeping in mind the arrow indicating the out direction.
\end{enumerate}

\section*{Conclusion}
Once the parachute is folded and in its deployment sleeve, it is ready to be installed in the airframe.

\end{document}
