\documentclass[]{auvsi_doc}
\setkeys{auvsi_doc.cls}{
	AUVSITitle={Unmanned Ground Vehicle Drive Test Procedures and Results},
	AUVSILogoPath={./figs/logo.pdf}
}

% include extra packages, if needed

\begin{document}

\begin{AUVSITitlePage}
\begin{artifacttable}
\entry{GV-004, 0.1, 10-26-2018, Initial creation\, procedures listed, Jacob Willis, <<Needs Review>>}

% additional \entry{} commands for extra rows in the revision table, if needed
\end{artifacttable}
\end{AUVSITitlePage}

% document contents (see below for LaTex commands that make your life easier)
\section{Introduction}
This document describes the procedures used to test each of the Unmanned Ground Vehicle (UGV) drive system.

\section {Test Procedures and Results}
%%%%%%%%%%%%%%%%%%%%%%%%%
	\subsection{Total Drive Mass}

	The total mass of all components that land on the ground and the vehicle must be capable of moving. Results are found in Table \ref{mass}.


	\begin{table}[!h]
	\centering
	
	\caption{Total mass for the UGV ground component.}
\label{mass}
	\begin{tabular}{|l|l|}
		\hline
		\rowcolor[HTML]{C0C0C0}
		\textbf{Concept}       & \textbf{Result} \\
		\hline
		RC Car Chassis              &                 \\
		Water bottle    &                 \\
		Control Board               &                 \\
		GPS                 & 			 \\
		Radio                 & 			 \\
		Parachute and Shrouds                 & 			 \\
		\hline
	\end{tabular}
	\end{table}


%%%%%%%%%%%%%%%%%%%%%%%%%
	\subsection{Maximum Drive Speed}

	
	The UGV will be driven 25 feet three times. The time it takes to drive that distance is averaged and used to calculate the maximum drive speed. To ensure that maximum speed
	is reached, the UGV will drive 25 feet prior to driving the timed distance.


	$$
	\frac{25}{\frac{2.16 + 2.00 + 1.97}{3}} = 12.23 \frac{\textrm{ft}}{\textrm{s}} = 8.33 \textrm{mph}
	$$

%%%%%%%%%%%%%%%%%%%%%%%%%
\subsection{Conclusion}
The preceding test results validate the current UGV system design. 

\end{document}
