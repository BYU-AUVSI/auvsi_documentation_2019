\documentclass[]{auvsi_doc}
\setkeys{auvsi_doc.cls}{
	AUVSITitle={Flight Test Log},
	AUVSIRevision=0.0,
	AUVSIDescription={Created},
	AUVSIAuthor={Kameron Eves},
	AUVSIChecker={[Checker]},
	AUVSILogoPath={./figs/logo.pdf},
	AUVSIDocID={AF-004}
}

% include extra packages, if needed
\usepackage{makecell}
\usepackage{longtable}

% Remove Heading Numbers
\setcounter{secnumdepth}{0}

\begin{document}
\begin{AUVSITitlePage}
\begin{artifacttable} 
	\entry{AF-004, 0.1, 11-07-2018, Created*, Kameron Eves, Andrew Torgesen}
	\entry{AF-004, 0.1, 2-5-2019, Added tracking of autonomous and total flight time, Kameron Eves, [Checker]}
\end{artifacttable}
*Note that additions to this log will not necessitate a revision update. Only formatting or other content additions will require that.
\end{AUVSITitlePage}
% document contents


\section{Log}

\begin{center}
%\begin{table}[h!]

\captionof{table}{A log of each flight test conducted by our team. Autonomous flight time is listed in bold under the total time.}
\bgroup
\def\arraystretch{1.25}%
\begin{longtable}{| c | p{2 cm} | c | p{9 cm} |}
	\hline 
	\makecell{\textbf{Date} \\ \textbf{(m-d-y)}}&\textbf{Location} &	\makecell{\textbf{Length}\\ Total \\ \textbf{Auto} \\ \textbf{(min)}} & 	\textbf{Notes} \\
	\hline\hline
	10-16-18 				& Springville			&	0.08					& 	Networking issues, later determined to be because of location. Moving down the road works. Attempted RC flight and crashed on launch. Need to practice launch procedure. \\
	\hline
	10-19-18 				& Springville			&	0.15 					& 	Attempted RC flight for imaging. RC lost upon launch. Later determined to be because of the RC antenna not being installed. Aircraft did not have a balanced CG and so performed a loop and crash landed. \\
	\hline
	10-23-18 				& Springville			&	1.28					& 	Attempted RC flight for imaging. Aircraft had major longitudinal stability issues. Later determined to be because of a negative static margin. Moving the batter forward fixes issue. Lost control and crashed. Transmitter also dying very very quickly. Later determined to be transmission power set to high (1 A changed to 10 mA). \\
	\hline
	11-01-18 				& Springville			&	2.83					& 	Attempted RC flight for imaging. Still had minor stability issues. We lost control and crash landed near end of flight. We later determined these issues were because the battery was not secured properly. It slid around inflight affecting our static margin. This caused instability and aggressive flight maneuvers that caused the battery to fall out in flight. As such we lost control and crashed. Battery must be strapped down.  \\
	\hline
	11-06-18 				& Springville			&	4.77					& 	Attempted RC flight for imaging. Aircraft flew wonderfully. Images of ground targets successfully captured. Flight was terminated when RC was lost and aircraft crashed hard. More investigation into the cause is needed. Possibly because of RC interference over the trees or to low of transmission power.  \\
	\hline
	12-11-18 				& Rock Canyon			&	0.97				& 	First flight test of new aircraft. Performed a glide test and found the aircraft performed well. With slight longitudinal instability. We moved the cg forward by adding a 500 g weight. This proved too much as the aircraft was notably nose heavy the next flight attempt which was forced to landed immediately upon takeoff . With an inexperienced pilot, we found this preferable to instability and attempted a second flight. The aircraft still dropped on takeoff but stable flight was obtained. This lasted 45 seconds before the pilot lost control and crashed. While pilot error likely played a part in this crash, we later determined that wind from the canyon was a large factor. We will avoid rock canyon in the future.\\
	\hline
	1-11-19 				& Rock Canyon			&	5.5				& First fully successful flight. Asked Alex Newell (an experienced RC pilot) to fly for us. Set trims. We also decided to add colored tape to the wings to increase the visibility of the aircraft in flight.\\
	\hline
	1-18-19 				& Rock Canyon			&	3.92				& Very windy, probably shouldn't have attempted flight but we had not yet figured out the wind problem from the canyon. Alex Newell flew again and decided to land for the safety of the aircraft. Adjusted trims.	\\
	\hline
	1-23-19 				& Rock Canyon			&	13.6				& First successful flight with our RC pilot flying. Performed two flights both of which were successful. Several minor repairs (general maintenance) were necessary after this flight. This flight test proved the aircraft was flyable and stable with all of the weight that will be on the aircraft during the competition. We tested both the weight with and without the UGV.\\
	\hline
	1-25-19				& Rock Canyon			&	10.75				& Trimmed aircraft with and without UGV weight.  Transferred these trims to ROSflight. 2 flights. Additional tape on wing provided sufficient visibility.	\\
	\hline
	1-31-19				& Utah County Airfield	&	7.00				& Intended to test autopilot and begin tuning gains. Could not successfully turn on autopilot. Later this was determined to be because we were incorrectly following the process to hand over control to the autopilot. We flew once manually to test the gains. Small adjustments were made and transferred to ROSflight. Alex Newell again acted as our safety pilot.	\\
	\hline
	2-2-19				& Utah County Airfield	&	\makecell{24.00 \\ \textbf{0.5}}				& Two flights to tune gains on autopilot. The autopilot had a tendency to flip the aircraft upside down immediately after turning on the autopilot. This was determined to not be caused by the gains. Aircraft landed safely manually both flights. After a couple of days of testing we found the cause was that the number being used to convert rad to PWM for the ailerons was negative. This effectively reversed the aileron polarity. Last year the wires must have been swapped from how we have them now. 	\\
	\hline
	2-6-19				& Utah County Airfield	&	\makecell{16.50 \\ \textbf{3.5}}				& Two flights to tune gains on autopilot. Flipping issue was fixed. Aircraft dove towards the ground upon turning on autopilot. This was fixed between flights (rad to PWM conversion for the elevator was negative this time) and we achieved autonomous flight the second attempt. We then tuned the longitudinal PID gains. Aircraft landed safely manually both flights.	\\
	\hline
\end{longtable}
\egroup
\label{table:Results}

\textbf{\underline{Statistics}}

%\end{table}
\end{center}

\textbf{Total Flight Time:} 1 Hour 31.5 Minutes 

\textbf{Total Autonomous Flight Time:} 4 Minutes

\textbf{Percent of Autonomous Flight:} 4.4\%

\textbf{Percentage of Flights Ending in Crash:} 33\%

\textbf{Percentage of Last 10 Flights Ending in Crash:} 0\%

\textbf{Flights Sense Last Crash:} 10


\end{document}