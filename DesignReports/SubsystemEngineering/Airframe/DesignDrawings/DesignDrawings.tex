\documentclass[]{auvsi_doc}
\setkeys{auvsi_doc.cls}{
	AUVSITitle={Airframe Subsystem Description},
	AUVSILogoPath={./../logo.pdf}
}

% include extra packages, if needed

\begin{document}

\begin{AUVSITitlePage}
\begin{artifacttable}
\entry{AF??, 0.1, 02-14-19, Initial Draft, Tyler Critchfield, CHECKED BY}
% additional \entry{} commands for extra rows in the revision table, if needed
\end{artifacttable}
\end{AUVSITitlePage}

% document contents (see below for LaTex commands that make your life easier)
\section{Introduction}
This artifact describes the final design of the Nimbus Pro airframe that was the chosen design concept in the Concept Development stage. Images of the built airframe are included, as well as a list of modifications we made to the plane. We then describe how this concept helps us achieve our key success measures.

\section{Design Description}
Our airframe is the Nimbus Pro aircraft we selected as our chosen design in the Concept Development stage. It is a fixed-wing plane with a large storage capacity and large wing span made of polystyrene (Fig \ref{fig:plane1}).

\begin{figure}[h!]
	\centering
	\includegraphics[width=.9\columnwidth]{plane1}
	\caption{Fully-constructed Nimbus Pro airframe before its first flight.}
	\label{fig:plane1}
\end{figure} 

The full procedure for constructing the plane will not be outlined here, but it follows basic principles and steps that are used in almost every aircraft construction:

\begin{itemize}
	\item Glue fuselage sections together
	\item Attach proper servos and control horns to each control surface
	\item Attach motors and ESCs to wings
	\item Attach tail pieces
	\item Install all electronic hardware/wiring
	\item Ensure connectors and joints are properly secured and strengthened if necessary
	\item The center of gravity (cg) was measured throughout construction to ensure it could be in its optimal location once all hardware was installed 
\end{itemize}

The following modifications were made to our airframe based on our application and integrated hardware:

\begin{itemize}
	\item Holes were made in the main compartment hatch for the RC and Ubiquiti antennas to poke through. There is space further bak in the plane for these components, but we wanted to move the center of gravity as far forward as possible to achieve a desireable static margin. They were also placed so as to not interfere with the GPS signal.
	\item The GPS slot on top was not used and taped over. The GPS was instead placed inside the nose of the plane to help with cg placement.
	\item The servos we used were a little larger than the original plane design anticipated. Because of this, some foam and plastic from servo locations needed to be cut away to allow them to fit.
	\item A tail wedge was inserted underneath the tail to increase the tail incidence angle (see further details below).
	\item Small holes were drilled into the wing connector pieces to allow for the large wires to extend from the ESCs in the wings to the fuselage.
\end{itemize}

In Concept Development, we decided that we would try to modify our plane by adding wing extensions to increase total span. This would help the plane fly slower by reducing induced drag. Flying at a lower velocity would help us achieve higher performance in our key success measures. In the process of building the plane and evaluating our timeline for this project, we've decided to forgo the wing extension modification. As described more in AF??, our airplane flies great without wing extensions, and any extra benefit to extensions would not be worth the time and effort required to design, implement and test these extensions.

One unanticipated modification we've made to the aircraft is the placement of a foam wedge underneath the tail of the plane. When modeling the Nimbus Pro, we were able to predict stable behavior and a high-performing aerodynamic efficiency (and thus a slower cruising velocity) for the airframe. These conclusions were predicated on a tail incidence angle of about -7 degrees. Upon getting the plane, we could not implement this angle without inserting a wedge beneath the tail to increase its incidence angle. At first, we decided to see if the plane did not need such a modification - it did fly, but had a consistent tendency to want to pitch forward. This made it longitudinally unstable and difficult to manually fly. After installing the tail wedge, this problem was averted. The plane is now stable and flies at the design speed we had originally planned for.

The tail wedge (see Fig \ref{fig:plane1ig:wedge}) is made out of extruded polystyrene, measured and cut manually in order to give the required incidence angle for the tail. Several notches were carved into the top for it to securely fit with the tail pieces. Fiber tape was then wrapped around it to ensure it would not come loose during flight.

\begin{figure}[h!]
	\centering
	\includegraphics[width=.75\columnwidth]{tailwedge}
	\caption{A foam tail wedge installed underneath the tail of the plane to increase its incidence angle and improve stability.}
	\label{fig:wedge}
\end{figure}


\end{document}
