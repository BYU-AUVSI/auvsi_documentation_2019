\documentclass[]{auvsi_doc}
\setkeys{auvsi_doc.cls}{
	AUVSITitle={Geolocation Algorithm Description},
	AUVSILogoPath={./../../../../_resources_/logo.pdf}
}

% include extra packages, if needed
\usepackage{parskip}
\usepackage[space]{grffile}
\usepackage{pdfpages}
\usepackage{hyperref}
\newcounter{includepdfpage}

\newcommand{\pdflinkdoc}[2]{
\includepdf[pages=-,link,offset=23mm -40mm,linkname=#1,pagecommand={\refstepcounter{includepdfpage}\label{#1.\theincludepdfpage}}]{#2}
}

\usepackage{listings}
\usepackage{color}

\definecolor{dkgreen}{rgb}{0,0.6,0}
\definecolor{gray}{rgb}{0.5,0.5,0.5}
\definecolor{mauve}{rgb}{0.58,0,0.82}

\begin{document}

\begin{AUVSITitlePage}
\begin{artifacttable}
\entry{IM-004, 1.0, 12-12-2018, Initial release, Connor Olsen, Tyler Miller}
\entry{IM-004, 1.0, 02-20-2019, Comment / code updates, Connor Olsen, Tyler Miller}
% additional \entry{} commands for extra rows in the revision table, if needed
\end{artifacttable}
\end{AUVSITitlePage}

\section{Introduction}

Geolocation of targets is one of our key success measures, whose accuracy is scored 
by the judges for points. Accuracy is determined by distance of our location estimate 
from ground truth, at a max of 150ft. Anything further than 150ft will score 0 points.
This geolocation algorithm is fast enough to work in real time on the server as 
targets are cropped. 

\section{Introduction}

Given how short the algorithm is (~150 lines), the best way to document it is 
through the code itself.

\lstinputlisting[language=Python]{./geolocation.py}

\end{document}
