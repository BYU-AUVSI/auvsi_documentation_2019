\documentclass[]{auvsi_doc}
\setkeys{auvsi_doc.cls}{
	AUVSITitle={Imaging Subsystem Description},
	AUVSILogoPath={./../../../../_resources_/logo.pdf}
}

% include extra packages, if needed
\usepackage{parskip}
\usepackage[space]{grffile}
\usepackage{pdfpages}
\usepackage{hyperref}
\newcounter{includepdfpage}

\newcommand{\pdflinkdoc}[2]{
\includepdf[pages=-,link,offset=25mm -30mm,linkname=#1,pagecommand={\refstepcounter{includepdfpage}\label{#1.\theincludepdfpage}}]{#2}
}

\begin{document}

\begin{AUVSITitlePage}
\begin{artifacttable}
\entry{IM-005, 1.0, 12-11-2018, Initial release, Tyler Miller \& Connor Olsen, Derek Knowles}
\entry{IM-005, 1.1, 02-14-2019, Update diagrams. Clarification, Tyler Miller, Connor Olsen}
% additional \entry{} commands for extra rows in the revision table, if needed
\end{artifacttable}
\end{AUVSITitlePage}

\section{Introduction}

Vision's imaging server will be the central interface between the raw data received from the plane 
the various ODLC (Object Detection Localization Classification) clients. Below is a brief overview of how it is setup and where its various components reside.

\section{Motivation}

This major code change was primarily motivated by using last years imaging
software and observing its shortcomings. Last years software required overly 
complex and difficult configuration and setup in order to produce actionable 
manual classifications. Manual classification  required 3-4 people, when it 
could be easily managed by 1 or 2. Finally, the system was fully dependent 
on ROS and was interconnected in such a way that it would be next to impossible to change
a single module without the entire package being affected.

Given these shortcomings we set out to create an imaging package that emphasized
the following:

1. Support up to N manual and autonomous imaging clients out of the box with no 
special configuration required.

2. Maximize modularity and transferability by separating the platform from ROS as 
much as possible.

3. Provide extensive documentation to ease future development and enable new developers
to learn quickly.

4. Build and release the code base in such a way that it can be run using a single command
on any platform.

\section{Server}

Once we fully understood last years imaging system we were able to lay out its weaknesses
and create the above objectives to best address them. One of our main concerns was how the 
current codebase spread its information across multiple machines: one machine would hold 
the raw images from the plane's camera, another would hold cropped images, and still another
would hold classification information. This made it very difficult to identify bugs, replicate
results and setup quickly. With these objectives and shortcomings in mind, a 
basic server-client architecture was defined as shown in Figure \ref{fig:serverFlow}.

\AUVSIFigure
{./figs/serverFlowchart.pdf}
{\textwidth}
{Server Architecture}
{fig:serverFlow}

Notice that the ROS Ingester is the only code block dependent on ROS. In reality, the
ingester is a simple 200 line python script which subscribes to the relavent ROS data
streams (aka: rostopics), and pushes their information into the database.

Model classes represent a single row in a table and essentially function as a translation
layer between the database and the top level ingester/Rest API. 

DAO's or Database Access Objects, interact directly with the database by encapsulating
SQL queries into methods that can be easily accessed and called by external modules.
These methods return model classes or require them as parameters.

For the database we chose Postgresql. Postgres is well known for its powerful
feature set, as well as easy out-of-the-box concurrency safety. The later was particularly
important in our application since there will often be multiple connections simultaneously 
querying most of the database.

Rest APIs are a ubiquitous interface and fit Vision's problem space well. By publishing a
Rest server and building our package around its ideology, support for up to N clients comes
naturally. 

The basic data flow of all these components is shown in Figure \ref{fig:dataFlow}.

\AUVSIFigure
{./figs/basicDataFlow.pdf}
{\textwidth}
{AUVSI Imaging Data Flow Through Autonomous/Manual Classification}
{fig:dataFlow}

Data flows back and forth between client and server, with the server holding a definitive-final
copy of a target image, as well as a history of its state during intermediate steps.

\textbf{Step 1} Takes all the data from the plane and pushes it into the database 
on the server. This is the "raw" data collected by the plane, which the server 
will store for the duration of the flight.

\textbf{Step 2} An autonomous or manual client requests the next raw image from the 
server. The server automatically keeps track of which images a client has or has not seen
and will only send new photos. The client then attempts to locate any targets, and will 
POST a cropped image to the server if it does.

\textbf{Step 3} An autonomous or manual client requests an unclassified cropped 
image. They can then proceed to classify it, identifying attributes such as shape,
letter, and color. All of these attributes are required by the AUVSI judges for 
maximum points. At this point the cropped image is also automatically run through 
a geolocation script (detailed more in IM-008) which attempts to pinpoint the 
targets latitude and longitude.

\textbf{Step 4} Often the plane will capture multiple images of the target, and thus 
multiple cropped images and classifications for that target will be generated by the 
previous steps. The server automatically bins similar classifications as a single 'target'.
In this step, users can choose which image and classification details will finally be submitted
to the judges (since we want to only submit a target once).

\textbf{Step 5} The server gets any targets declared ready for submission and sends
them to the interop subsystem. Interop is in charge of interactions between the 
judge server and all subsystems and will handle submitting the final targets.

The manual client GUI codebase is detailed in depth in IM-001 and IM-002. Details
on the geolocation algorithm can be found in IM-008, while a description of the server's
API can be found in IM-003.

\section{Conclusion}

Properly implemented, the server infrastructure allows simple one click installation
for anyone looking to run imaging. Coupled with the vision team's emphasis on comprehensive
documentation and official code releases, the imaging codebase will be as transferable as 
possible for future AUVSI team members.

Besides this high degree of transferability, this design is also modular. Different 
portions of the codebase could be re-implemented or changed with little to no effect on 
other code modules. The strengths of the new imaging model respond to some of the largest
issues in the old imaging codebase and create a reliable, easy to use, modular solution
to AUVSI's imaging problem

\end{document}
