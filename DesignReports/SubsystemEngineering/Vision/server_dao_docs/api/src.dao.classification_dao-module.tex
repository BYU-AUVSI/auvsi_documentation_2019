%
% API Documentation for AUVSI Imaging Server
% Module src.dao.classification_dao
%
% Generated by epydoc 3.0.1
% [Mon Dec 10 18:31:41 2018]
%

%%%%%%%%%%%%%%%%%%%%%%%%%%%%%%%%%%%%%%%%%%%%%%%%%%%%%%%%%%%%%%%%%%%%%%%%%%%
%%                          Module Description                           %%
%%%%%%%%%%%%%%%%%%%%%%%%%%%%%%%%%%%%%%%%%%%%%%%%%%%%%%%%%%%%%%%%%%%%%%%%%%%

    \index{src \textit{(package)}!src.dao \textit{(package)}!src.dao.classification\_dao \textit{(module)}|(}
\section{Module src.dao.classification\_dao}

    \label{src:dao:classification_dao}

%%%%%%%%%%%%%%%%%%%%%%%%%%%%%%%%%%%%%%%%%%%%%%%%%%%%%%%%%%%%%%%%%%%%%%%%%%%
%%                           Class Description                           %%
%%%%%%%%%%%%%%%%%%%%%%%%%%%%%%%%%%%%%%%%%%%%%%%%%%%%%%%%%%%%%%%%%%%%%%%%%%%

    \index{src \textit{(package)}!src.dao \textit{(package)}!src.dao.classification\_dao \textit{(module)}!src.dao.classification\_dao.ClassificationDAO \textit{(class)}|(}
\subsection{Class ClassificationDAO}

    \label{src:dao:classification_dao:ClassificationDAO}
\begin{tabular}{cccccccc}
% Line for object, linespec=[False, False]
\multicolumn{2}{r}{\settowidth{\BCL}{object}\multirow{2}{\BCL}{object}}
&&
&&
  \\\cline{3-3}
  &&\multicolumn{1}{c|}{}
&&
&&
  \\
% Line for src.dao.base\_dao.BaseDAO, linespec=[False]
\multicolumn{4}{r}{\settowidth{\BCL}{src.dao.base\_dao.BaseDAO}\multirow{2}{\BCL}{src.dao.base\_dao.BaseDAO}}
&&
  \\\cline{5-5}
  &&&&\multicolumn{1}{c|}{}
&&
  \\
&&&&\multicolumn{2}{l}{\textbf{src.dao.classification\_dao.ClassificationDAO}}
\end{tabular}

Does most of the heavy lifting for classification tables: 
outgoing\_autonomous and outgoing\_manual. Contains general database 
methods which work for both types.


%%%%%%%%%%%%%%%%%%%%%%%%%%%%%%%%%%%%%%%%%%%%%%%%%%%%%%%%%%%%%%%%%%%%%%%%%%%
%%                                Methods                                %%
%%%%%%%%%%%%%%%%%%%%%%%%%%%%%%%%%%%%%%%%%%%%%%%%%%%%%%%%%%%%%%%%%%%%%%%%%%%

  \subsubsection{Methods}

    \vspace{0.5ex}

\hspace{.8\funcindent}\begin{boxedminipage}{\funcwidth}

    \raggedright \textbf{\_\_init\_\_}(\textit{self}, \textit{configFilePath}, \textit{outgoingTableName})

\setlength{\parskip}{2ex}
    Startup the DAO. Attempts to connect to the postgresql database using 
    the settings specified in the confg.ini file

\setlength{\parskip}{1ex}
      Overrides: object.\_\_init\_\_ 	extit{(inherited documentation)}

    \end{boxedminipage}

    \label{src:dao:classification_dao:ClassificationDAO:upsertClassification}
    \index{src \textit{(package)}!src.dao \textit{(package)}!src.dao.classification\_dao \textit{(module)}!src.dao.classification\_dao.ClassificationDAO \textit{(class)}!src.dao.classification\_dao.ClassificationDAO.upsertClassification \textit{(method)}}

    \vspace{0.5ex}

\hspace{.8\funcindent}\begin{boxedminipage}{\funcwidth}

    \raggedright \textbf{upsertClassification}(\textit{self}, \textit{classification})

    \vspace{-1.5ex}

    \rule{\textwidth}{0.5\fboxrule}
\setlength{\parskip}{2ex}
    Upserts a classification record. If the image\_id given in the 
    classification object already exists within the table, the 
    corresponding record is updated. If it doesn't exist, then we insert a 
    new record.

\setlength{\parskip}{1ex}
      \textbf{Parameters}
      \vspace{-1ex}

      \begin{quote}
        \begin{Ventry}{xxxxxxxxxxxxxx}

          \item[classification]

          The outgoing\_autonomous or manual classification to upsert. Note
          that these objects do not require all classification properties 
          to be successfully upserted. At a minimum it must have image\_id.
          ie: upsert could work if you provided a classification object 
          with only image\_id, shape and shape\_color attributes.

            {\it (type=outgoing\_manual)}

        \end{Ventry}

      \end{quote}

      \textbf{Return Value}
    \vspace{-1ex}

      \begin{quote}
      The resulting table id (Note: not image\_id) of the classification 
      row if successfully upserted, otherwise -1

      {\it (type=int)}

      \end{quote}

    \end{boxedminipage}

    \label{src:dao:classification_dao:ClassificationDAO:addClassification}
    \index{src \textit{(package)}!src.dao \textit{(package)}!src.dao.classification\_dao \textit{(module)}!src.dao.classification\_dao.ClassificationDAO \textit{(class)}!src.dao.classification\_dao.ClassificationDAO.addClassification \textit{(method)}}

    \vspace{0.5ex}

\hspace{.8\funcindent}\begin{boxedminipage}{\funcwidth}

    \raggedright \textbf{addClassification}(\textit{self}, \textit{classification})

    \vspace{-1.5ex}

    \rule{\textwidth}{0.5\fboxrule}
\setlength{\parskip}{2ex}
    Adds the specified classification information to one of the outgoing 
    tables

\setlength{\parskip}{1ex}
      \textbf{Parameters}
      \vspace{-1ex}

      \begin{quote}
        \begin{Ventry}{xxxxxxxxxxxxxx}

          \item[classification]

          The classifications to add to the database

            {\it (type=outgoing\_autonomous or outgoing\_manual)}

        \end{Ventry}

      \end{quote}

      \textbf{Return Value}
    \vspace{-1ex}

      \begin{quote}
      Id of classification if inserted, otherwise -1

      {\it (type=int)}

      \end{quote}

    \end{boxedminipage}

    \label{src:dao:classification_dao:ClassificationDAO:getClassificationByUID}
    \index{src \textit{(package)}!src.dao \textit{(package)}!src.dao.classification\_dao \textit{(module)}!src.dao.classification\_dao.ClassificationDAO \textit{(class)}!src.dao.classification\_dao.ClassificationDAO.getClassificationByUID \textit{(method)}}

    \vspace{0.5ex}

\hspace{.8\funcindent}\begin{boxedminipage}{\funcwidth}

    \raggedright \textbf{getClassificationByUID}(\textit{self}, \textit{id})

    \vspace{-1.5ex}

    \rule{\textwidth}{0.5\fboxrule}
\setlength{\parskip}{2ex}
    Attempts to get the classification with the specified 
    universal-identifier

\setlength{\parskip}{1ex}
      \textbf{Parameters}
      \vspace{-1ex}

      \begin{quote}
        \begin{Ventry}{xx}

          \item[id]

          The id of the image to try and retrieve

            {\it (type=int)}

        \end{Ventry}

      \end{quote}

    \end{boxedminipage}

    \label{src:dao:classification_dao:ClassificationDAO:getClassification}
    \index{src \textit{(package)}!src.dao \textit{(package)}!src.dao.classification\_dao \textit{(module)}!src.dao.classification\_dao.ClassificationDAO \textit{(class)}!src.dao.classification\_dao.ClassificationDAO.getClassification \textit{(method)}}

    \vspace{0.5ex}

\hspace{.8\funcindent}\begin{boxedminipage}{\funcwidth}

    \raggedright \textbf{getClassification}(\textit{self}, \textit{id})

    \vspace{-1.5ex}

    \rule{\textwidth}{0.5\fboxrule}
\setlength{\parskip}{2ex}
    Gets a classification by the TABLE ID. This is opposed to 
    getClassificationByUID, which retrieves a row based off of the unique 
    image\_id This is mostly used internally, and is not used by any of the
    public REST API methods.

\setlength{\parskip}{1ex}
      \textbf{Parameters}
      \vspace{-1ex}

      \begin{quote}
        \begin{Ventry}{xx}

          \item[id]

          The table id of the classification to retrieve.

            {\it (type=int)}

        \end{Ventry}

      \end{quote}

      \textbf{Return Value}
    \vspace{-1ex}

      \begin{quote}
      String list of values retrieved from the database. Child classes will
      properly place these values in model objects. If the given id doesn't
      exist, None is returned.

      {\it (type=[string])}

      \end{quote}

    \end{boxedminipage}

    \label{src:dao:classification_dao:ClassificationDAO:getAll}
    \index{src \textit{(package)}!src.dao \textit{(package)}!src.dao.classification\_dao \textit{(module)}!src.dao.classification\_dao.ClassificationDAO \textit{(class)}!src.dao.classification\_dao.ClassificationDAO.getAll \textit{(method)}}

    \vspace{0.5ex}

\hspace{.8\funcindent}\begin{boxedminipage}{\funcwidth}

    \raggedright \textbf{getAll}(\textit{self})

    \vspace{-1.5ex}

    \rule{\textwidth}{0.5\fboxrule}
\setlength{\parskip}{2ex}
    Get all the images currently in this table

\setlength{\parskip}{1ex}
      \textbf{Return Value}
    \vspace{-1ex}

      \begin{quote}
      A cursor to the query result for the specified classification type. 
      This allows children classes to place the results in their desired 
      object type.

      {\it (type=cursor)}

      \end{quote}

    \end{boxedminipage}

    \label{src:dao:classification_dao:ClassificationDAO:updateClassificationByUID}
    \index{src \textit{(package)}!src.dao \textit{(package)}!src.dao.classification\_dao \textit{(module)}!src.dao.classification\_dao.ClassificationDAO \textit{(class)}!src.dao.classification\_dao.ClassificationDAO.updateClassificationByUID \textit{(method)}}

    \vspace{0.5ex}

\hspace{.8\funcindent}\begin{boxedminipage}{\funcwidth}

    \raggedright \textbf{updateClassificationByUID}(\textit{self}, \textit{id}, \textit{updateClass})

    \vspace{-1.5ex}

    \rule{\textwidth}{0.5\fboxrule}
\setlength{\parskip}{2ex}
    Builds an update string based on the available key-value pairs in the 
    given classification object if successful, returns an classification 
    object of the entire row that was updated

\setlength{\parskip}{1ex}
      \textbf{Parameters}
      \vspace{-1ex}

      \begin{quote}
        \begin{Ventry}{xxxxxxxxxxx}

          \item[id]

          The image\_id of the classification to update

            {\it (type=int)}

          \item[updateClass]

          Information to attempt to update for the classification with the 
          provided image\_id

            {\it (type=outgoing\_autonomous or outgoing\_manual)}

        \end{Ventry}

      \end{quote}

      \textbf{Return Value}
    \vspace{-1ex}

      \begin{quote}
      The classification of the now updated image\_id if successful. 
      Otherwise None

      {\it (type=outgoing\_autonomous or outgoing\_manual)}

      \end{quote}

    \end{boxedminipage}

    \label{src:dao:classification_dao:ClassificationDAO:getAllDistinct}
    \index{src \textit{(package)}!src.dao \textit{(package)}!src.dao.classification\_dao \textit{(module)}!src.dao.classification\_dao.ClassificationDAO \textit{(class)}!src.dao.classification\_dao.ClassificationDAO.getAllDistinct \textit{(method)}}

    \vspace{0.5ex}

\hspace{.8\funcindent}\begin{boxedminipage}{\funcwidth}

    \raggedright \textbf{getAllDistinct}(\textit{self}, \textit{modelGenerator}, \textit{whereClause}={\tt None})

    \vspace{-1.5ex}

    \rule{\textwidth}{0.5\fboxrule}
\setlength{\parskip}{2ex}
    Get all the unique classifications in the classification queue 
    Submitted or not.

\setlength{\parskip}{1ex}
    \end{boxedminipage}


\large{\textbf{\textit{Inherited from src.dao.base\_dao.BaseDAO\textit{(Section \ref{src:dao:base_dao:BaseDAO})}}}}

\begin{quote}
basicTopSelect(), close(), conn(), executeStatements(), getResultingId()
\end{quote}

\large{\textbf{\textit{Inherited from object}}}

\begin{quote}
\_\_delattr\_\_(), \_\_format\_\_(), \_\_getattribute\_\_(), \_\_hash\_\_(), \_\_new\_\_(), \_\_reduce\_\_(), \_\_reduce\_ex\_\_(), \_\_repr\_\_(), \_\_setattr\_\_(), \_\_sizeof\_\_(), \_\_str\_\_(), \_\_subclasshook\_\_()
\end{quote}

%%%%%%%%%%%%%%%%%%%%%%%%%%%%%%%%%%%%%%%%%%%%%%%%%%%%%%%%%%%%%%%%%%%%%%%%%%%
%%                              Properties                               %%
%%%%%%%%%%%%%%%%%%%%%%%%%%%%%%%%%%%%%%%%%%%%%%%%%%%%%%%%%%%%%%%%%%%%%%%%%%%

  \subsubsection{Properties}

    \vspace{-1cm}
\hspace{\varindent}\begin{longtable}{|p{\varnamewidth}|p{\vardescrwidth}|l}
\cline{1-2}
\cline{1-2} \centering \textbf{Name} & \centering \textbf{Description}& \\
\cline{1-2}
\endhead\cline{1-2}\multicolumn{3}{r}{\small\textit{continued on next page}}\\\endfoot\cline{1-2}
\endlastfoot\multicolumn{2}{|l|}{\textit{Inherited from object}}\\
\multicolumn{2}{|p{\varwidth}|}{\raggedright \_\_class\_\_}\\
\cline{1-2}
\end{longtable}

    \index{src \textit{(package)}!src.dao \textit{(package)}!src.dao.classification\_dao \textit{(module)}!src.dao.classification\_dao.ClassificationDAO \textit{(class)}|)}
    \index{src \textit{(package)}!src.dao \textit{(package)}!src.dao.classification\_dao \textit{(module)}|)}
